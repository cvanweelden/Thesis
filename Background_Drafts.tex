\section{Background}


%learning problems notation, \mathcal{X}, vectors in Euclidean space

%Euclidean distances

%Mahalanobis distance box

%Parametric distance

%Constraint types

%Optimization, solving in A vs in L, constrained (A PSD) vs unconstrained

%Regularization

%Fixed rank learning


\section*{Exploratory draft 1}

Metric learning methods learn a distance function in a supervised manner. Supervised learning requires a training set of data points and some sort of side information about those points which can inform the learning process. In practice, this side information comes in the form of class labels for each point in classification problems, relative distance constraints in ranking problems, and the correct structured output in structured prediction problems. This side information is often referred to as the ground-truth.

Training points are assumed to be $d$-dimensional real-valued feature vectors in a Euclidean feature space $\mathcal{X} = \mathbb{R}^d$. Let $\mat{X} = \left[ \vec{x}_1, \vec{x}_2, ..., \vec{x}_n \right]$ be the matrix of all the training points, with each column being a single feature vector $\vec{x}_i \in \mathcal{X}$. Each element of a training vector corresponds to a single feature value. Distances between points in the feature space can be measured by the Euclidean distance $\|\vec{x}_i - \vec{x}_j\|_2 = \sqrt{(\vec{x}_i - \vec{x}_j)^T(\vec{x}_i - \vec{x}_j)}$, although in practice the square root is often dropped because the squared distance is easier to compute.



\subsection{Parametric distance functions}

The goal of metric learning is to learn a distance function in the feature space. The learned distance function is a function $d: \mathcal{X} \times \mathcal{X} \rightarrow \mathbb{R}^{+}$ having metric or pseudo-metric properties:
\begin{subequations}
\begin{align}
        d(\vec{x}_i, \vec{x}_j) & \geq 0 &  \text{non-negativity}, \label{eq:non-negativity}\\
        d(\vec{x}_i, \vec{x}_i) & = 0 &  \text{identity}, \label{eq:identity}\\
        d(\vec{x}_i, \vec{x}_j) & = d(\vec{x}_j, \vec{x}_i) &  \text{symmetry}, \label{eq:symmetry}\\
        d(\vec{x}_i, \vec{x}_k) & \leq d(\vec{x}_i, \vec{x}_j) + d(\vec{x}_j, \vec{x}_k) &  \text{triangle unequality}. \label{eq:triangle_unequality}
\end{align}
\end{subequations}
The properties of a metric are the same except that (\ref{eq:identity}) is replaced by identity of indiscernibles: $d(\vec{x}_i, \vec{x}_j) = 0 \xleftrightarrow[]{\text{\tiny{iff}}} i = j$. The distance function is often parameterized by a $d \times d$ matrix $\mat{A}$ in the following way:
\begin{equation}
d_{\mat{A}}(\vec{x}_i, \vec{x}_j) = \sqrt{(\vec{x}_i - \vec{x}_j)^T\mat{A}(\vec{x}_i - \vec{x}_j)}.
\end{equation}
In the metric learning literature this form of distance function is often called a Mahalanobis distance, although this term was originally used for a specific instance of this distance, see the text box. If $\mat{A}$ is positive definite then $d_{\mat{A}}$ is a metric and if $\mat{A}$ is positive semi-definite then $d_{\mat{A}}$ is a pseudo-metric. If $\mat{A}$ equals the identity matrix then $d_{\mat{A}}$ is simply the Euclidean distance.

















\section*{Old snippets}


\subsection{Supervised metric learning}
\label{sec:metric_learning}



Many machine learning methods depend on pairwise comparisons between object representations. 

Euclidean distance does not always correspond with semantic distance or similarity.

The aim of metric learning methods is to learn a distance function helpful to the problem.

%local vs global, supervised vs unsupervised, linear vs non-linear



\subsubsection{Notation and problem formulation}


The problem of metric learning is to learn the parameters for a distance function from training examples.

A metric or distance function is a function $d: \mathcal{X} \times \mathcal{X} \rightarrow \mathbb{R}^{+}$, having metric or semi-metric properties (look these up).

The Mahalanobis metric is a linear parametric distance function and is most commonly used in metric learning.

% Linear metrics can be interpreted as a linear transformation on the input space.

% Although some methods learn 

structured prediction, predict complex outputs, sequences or trees, combinatorially many possible outputs, 0-1 loss not applicable, metric 

%need to parameterize the distance function, need some form of training data
%parametric distance metric: Mahalanobis distance, changing the matrix $M$ changes the distance metric, we can learn $M$ such that the distance satisfies our constraints
%Suppose we have a pair of vectors $\vec{x}, \vec{y} \in \mathbb{R}^D$
%\begin{equation}
%d_{M}(\vec{x},\vec{y}) = \sqrt{(\vec{x}-\vec{y})^{T} \mat{M} (\vec{x}-\vec{y})}
%\label{eq:mahalanobis_distance}
%\end{equation}
%metric: identity of indiscernibles: d(x,y) = 0 => x = y, symmetric: d(x,y) = d(y,x), non-negative: d(x,y) >= 0, sub-additive (triangle inequality): d(x,y) + d(y,z) >= d(x,z)
%pseudo: relaxes identity of indiscernibles to: d(x,x) = 0, meaning that there might be y not equal to x such that d(x,y) = 0
%Mahalanobis distance is pseudo-metric if $M$ is positive semi-definite and metric if $M$ is positive definite
%Linear metric, can be rewritten as linear transformation of the input space, comes in handy for nearest neighbor retrieval since we can still use Euclidean distance between transformed points
%rewrite $M = L^T L$, learn L, M is always PSD and the metric thus a pseudo metric, L is a linear transformation of the feature space such that Euclidean norm of transformed difference vector (or difference between transformed vectors) corresponds to $d_M$
%\begin{equation}
%d_{M}(\vec{x},\vec{y}) = ||\mat{L}(\vec{x}_{i}-\vec{y}_{i})||
%\label{eq:mahalanobis_transformation}
%\end{equation}
%$||\mat{L}(\vec{x}_{i}-\vec{y}_{i})|| = ||\mat{L}\vec{x}_{i}-\mat{L}\vec{y}_{i}||$


\subsubsection{Learning from equivalence constraints}

Suppose we have a problem with discrete labels for the examples, for classification or clustering we would like distances between examples to be small if their labels are the same and large otherwise.

\todo{Describe these methods (related work)}

%suppose we have a problem with input space $\mathcal{X} \subseteq \mathbb{R}^n$ and as output discrete class labels $\mathcal{Y} = {1, 2, ..., C}$. Then we want distance to be small iff labels are the same. kNN example, clustering example
%training set: subsets or pairs of objects which are similar. Many methods also use dissimilarity information: subsets or pairs of objects which are dissimiilar.
%try to make within-class variance small, difference between different classes large
%very applicable for classification tasks, but not so much for non-binary task such as those with a structured loss function, have to select some arbitrary cut-off for similarity or dissimilarity

The \acf{ITML} method \cite{davis2007information} considers similarity constraints of the form $d_{\mat{M}}(\vec{x}_{i}, \vec{x}_{j}) \leq u$ and dissimilarity constraints of the form $d_{\mat{M}}(\vec{x}_{i}, \vec{x}_{k}) \geq \ell$, where $u$ and $\ell$ are numeric upper and lower bounds on the distance between similar points pairs $\vec{x}_{i}, \vec{x}_{j}$ and dissimilar point pairs $\vec{x}_{i}, \vec{x}_{k}$. A matrix $\mat{M}$ is learned that satisfies these constraints while minimizing LogDet divergence between an input matrix $\mat{M}_0$ and $\mat{M}$.


$d_{\mat{M}}(\vec{x}_{i}, \vec{x}_{j}) + 1 \leq d_{\mat{M}}(\vec{x}_{i}, \vec{x}_{k})$
\subsubsection{Learning from ranking constraints}

Suppose we have a problem without discrete labels but in which we try to find the most similar examples, we would like the distance to a similar example to be smaller than the distance to a less similar example.

$d_{\mat{M}}(\vec{x}_{i}, \vec{x}_{j}) < d_{\mat{M}}(\vec{x}_{i}, \vec{x}_{k})$

\subsubsection{Learning from distance constraints}

Metric learning method generally don't do this, however MDS and visualization methods seem related in that they learn from pairwise distance matrix

\subsection{Nearest neighbor classification}

\subsection{Semantic segmentation}

%training set: ranking or triples of objects for which one is closer to the other
%try to satisfy as many of these constraints as possible


\subsection{SGD}
 \cite{rakhlin2012making} ASGD with suffix averaging is O(1/T) \cite{bottou2010large} ASGD or SGDQN is good for large scale problems \cite{bottou2008tradeoffs} SGD algorithms work surprisingly well in large scale learning \cite{xu2011towards} ASGD with specific parameters is optimal



