
\section*{Short description of graduation project.}

Metric learning is the problem of learning a distance function between data samples which satisfies some given constraints. Metric learning methods are useful in machine learning problems where the decision rule is based on distances between points in some numerical feature space, such as in the $k$-nearest neighbor classifier. For a $k$NN classifier we would like to learn a metric in which distances between points of different classes are large, while distances between points of the same class are small.

Most metric learning methods depend on samples being assigned to disjunct classes in order to generate constraints from which to learn the distance function. These methods are applicable to learning problems characterized by 0/1-loss. However, in some learning problems the quantity of interest is a distance measure which represents some form of similarity or difference between data points, e.g. age difference between people or semantic overlap between images. Similarly, in structured prediction problems the loss function is real-valued, since the space of possible predictions is is very large or infinite and some of these predictions are closer to the truth than others, making 0/1-loss uninformative. Metric learning methods that depend on class-equivalence constraints are not applicable to these problems.

This thesis investigates how we can apply metric learning methods to learning problems which have a real-valued loss function, such as structured prediction problems. We propose a method which views metric learning as a regression problem in the space of semi-definite matrices to learn a Mahalanobis distance which satisfies real-valued constraints. Satisfying these constraints aligns the feature space to the ground-truth space in which the loss function is measure, hence we refer to our method as \emph{metric alignment}. We apply this method to images taken from a semantic segmentation problem in which we learn to predict semantic overlap between image patches without knowing the correct segmentation, and to images from an attribute-based classification problem where we learn to predict an attribute vector from the image features. We compare our methods against existing methods using 0/1-loss based binary similarity constraints. The main contributions of this thesis are:
\begin{enumerate}
\item We describe a method for applying metric learning to problems characterized by a real-valued loss function.
\item We introduce two learning problems based on existing datasets that can be used to evaluate metric learning methods for real-valued loss functions.
\item We evaluate our method against existing methods on these learning problems.
\end{enumerate}

\end{document}