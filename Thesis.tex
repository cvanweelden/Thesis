\documentclass[a4paper,titlepage]{article}
\usepackage{listings}
\usepackage{graphicx}
\usepackage[font=small,labelfont=bf]{caption}
\usepackage{subcaption}
%\usepackage[scale=.65]{geometry}
%\usepackage{fullpage}
%\usepackage{a4wide}
\usepackage{amsmath}
\usepackage{mathtools}
\usepackage{float}
\usepackage{acronym}
\usepackage{cite}
\usepackage{url}
\usepackage[usenames, pdftex]{color}
\usepackage{soul}
\usepackage{rotating}
\usepackage{booktabs}
\usepackage{todonotes}
\usepackage{amsfonts}
\usepackage{mdframed}
\usepackage{algorithmic}
\usepackage{algorithm}

\renewcommand{\algorithmicrequire}{\textbf{Input:}}
\renewcommand{\algorithmicensure}{\textbf{Output:}}

\setlength{\marginparwidth}{4cm} %else todo notes will only fill halve the margin

\newcommand{\note}[1]{\marginpar{\colorbox{yellow}{\parbox{3.7cm}{#1}}}}

\renewcommand{\vec}[1]{\mathbf{#1}}
\newcommand{\mat}[1]{\mathbf{#1}}
\DeclareMathOperator*{\argmax}{arg\,max}
\DeclareMathOperator*{\argmin}{arg\,min}

\acrodef{MDS}{multidimensional scaling}
\acrodef{MSE}{mean squared error}
\acrodef{LMNN}{large margin nearest neighbor}
\acrodef{kNN}[$k$NN]{$k$-nearest neighbors}
\acrodef{HCRF}{hierarchical conditional random field}
\acrodef{SGD}{stochastic gradient descent}
\acrodef{ITML}{information-theoretic metric learning}

%\title{Metric alignment}
%\title{Aligning metric to a distance function in an unknown space.}
%\title{Metric alignment: learning distance functions that linearly approximate arbitrary similarity measures.}
\title{Metric alignment: linear approximation of arbitrary similarity measures.}
%\title{}
%\title{}
%\title{}
%\title{}


\author{Carsten van Weelden  \\ \texttt{cvanweelden@gmail.com} \\ 0518824 \and \small{Supervisor:} \\ Jan van Gemert \\ \texttt{J.C.vanGemert@uva.nl} \and University of Amsterdam\\
  The Netherlands \\ \\ In partial fulfillment etc.}
%\date{}                                           % Activate to display a given date or no date

\begin{document}
\maketitle

\begin{abstract}

\end{abstract}

\pagebreak

\tableofcontents

\pagebreak


%Boxes to write:
% kNN classification
% k-means clustering
% structured prediction problems (or)
% loss functions used in machine learning
% image segmentation problems
% zero-shot learning / attribute based learning





%3.5(bad) student network example
%3.7 proposed method should be used when you need a distance function measuring similarity or a feature space in which Euclidean distances correspond to similarity between objects for which it is difficult to decide whether they belong to the same discrete class and the similarity you want to measure is known for at least a subset of the training set but not for new examples but training and test objects can be represented by the same set of features.

%4.1 metric alignment, continuous distance functions, unknown or unavailable during execution, transforms feature space, feature distances are predictive of training distance
%3.4-2 metric alignment, learn a metric that satisfies real similarity score



%dimension reduction and visualization methods, (multi-dimensional scaling, isomap, LLE, SNE), mapping/embedding retaining pairwise distances, finding coordinates in some space which satisfy a similarity score



\section{Introduction.}

%2.1 objects represented as vectors of feature values, feature space
%1.1-1 machine learning features, feature space,
In machine learning tasks, objects of interest are represented using a set of features. Features correspond to attributes of the object that can be encoded in some way. For example the features representing images may simply be the pixel values of an image, but may also be more complex features computed from the basic pixel values. Each object is thus represented by a feature vector containing the feature values. We will refer to the space of all possible feature vectors as the \emph{feature space} and in the case of $d$-dimensional real-valued vectors take this space to be $\mathbb{R}^d$. \todo{Too abstract, might loose some readers in the first paragraph. Too fast into the mathematical notation. Maybe start with application. Show an image of the example (maybe replace cars with the animals set). Try to also show ground truth metric in an image.}

%2.2-1 measure similarity by assuming small difference in feature vector means similar objects,
%1.1-2 measuring distance, finding similar samples, Euclidean distance
%computing distance between data samples fundamental problem
%3.1-1 similarity between objects, distance function,
Now suppose we would like to measure similarities between objects. For example, we have some images of a Ferrari car and we want to find images of similar cars. Assuming that similar objects result in similar feature values, we can express similarity between objects as a distance function in feature space. For our $\mathbb{R}^d$ feature space we can therefore measure similarity by Euclidean distance between the points in the feature space given by the feature vectors. Thus, starting from the feature representations of our input images we would expect to find images of similar cars nearer in feature space than images of dissimilar cars.

%3.2 representation of objects, feature space, distance in this space different from similarity, explain problem with clustering: picking different features or scaling will lead to different clusters which do not all correspond to our idea of similarity, common to use Euclidean or Hamming distance depending on the feature vector
%2.3-1 feature difference not always good measure for similarity,
However, Euclidean distance is not necessarily the best measure for similarity. In most cases features are predetermined since they measure basic attributes of the object, but the type of similarity we want to measure is specific to the task. With Euclidean distance each feature has an equal impact on the distance, even though some features are much more relevant to the type of similarity we want to measure than others. For example in our car images example, we would want to focus on similarities in color if we are simply looking for other red cars, but we would want to look at more complex similarities in shape if we are looking for other racing cars. 

%1.2 use of metric in machine learning methods, kNN, k-means, SVM, effectiveness depends on metric
%2.2-2 used by kNN and k-means
%3.1-2 applications (clustering, non-parametric methods based on distance or local density [kernel density estimation, k-nearest neighbor classification], visual identification)
Several machine learning methods use a distance function at the core of their algorithm. For example, the \ac{kNN} algorithm classifies objects by searching the training data for feature vectors that are closest to the input vector. The objects corresponding to these nearest neighbors then vote on the class to assign to the input object. It is clear that for this to be effective, distances in feature space should be small for objects of the same class and large for objects of differing class. Similarly, the $k$-means clustering algorithm clusters samples based on their distances in feature space. This will cluster similar objects together if they are close together in feature space. These methods require us to carefully choose a feature representation and distance function in order to be effective.

%1.3-1 metric learning, parametric distance function, learn parameters,
%2.3-2 metric learning: learn distance function that is more useful
%3.3-1 metric learning: learn a distance function between object representations that corresponds to actual similarity, metric, parametric metrics, metric transform, Mahalanobis distance,
\todo{Could sell this as: Since we have some background knowledge about what we want the distance function to be like, we can use this knowledge to automatically learn ...}
Instead of hand-crafting a distance function, we can automatically learn a suitable distance function in a given feature space. This approach is called \emph{metric learning}. Metric learning methods define a parametric distance function and then learn the best parameters for this function. Learning the parameters requires a set of training samples in the given feature space and a set of constraints on the distances between training samples. Parameters of the distance function are learned that best fit these distance constraints. The constraints thus represent the similarity measure specific to our task. For example, if we are still looking for images of similar racing cars, we would provide samples of racing cars and then constrain the distances between these to be small, while constraining the distances to other types of cars to be large.

%1.3-2 constraints based on class of samples
%2.4-1 train metric on class differences,
%3.3-2 using discrete similarity for training or a similarity ranking,
Metric learning has been researched most in the context of multi-class classification problems and ranking problems. Because of this background, the constraints generally take the form of similarity/dissimilarity constraints or relative distance constraints. Similarity and dissimilarity constraints are generated from class labels: objects of the same class are constrained to have a small pairwise distance while objects of different classes are constrained to have a larger distance. In contrast, relative distance constraints are used for tasks where it might be difficult to make absolute similarity judgments, but where we can identify one object to be more similar to a given object than some other object. Relative distance constraints constrain one pairwise distance to be smaller than another pairwise distance.

%3.4-1 what if we have exact similarity scores represented as real numbers, but not discrete classes, e.g. structured problem
%research questions: how can metric learning be applied to real-valued loss, how does this compare to existing problems
However, there is a more direct approach to the metric learning problem: instead of generating similarity/dissimilarity or relative distance constraints, a metric can be trained on absolute distance constraints that directly specify what the distance between a given pair of points in feature space should be. \todo{Put more emphasis on target metric. Next to having labels we also have an exact idea of what the metric should look like. We differentiate from other methods in that we can use a target metric and align to that, while other methods can not do this.} Absolute distance constraints make more fine-grained similarity judgments than the binary similarity/dissimilarity constraints while supplying stronger supervision than relative distance constraints. Despite these advantages, absolute distance constraints have received little attention in the metric learning literature. Therefore, this thesis investigates the use of absolute distance constraints for metric learning. In doing so it answers the following research questions:
\begin{itemize}
\item How can we apply metric learning methods to problems with absolute distance constraints?
\item How does metric learning with absolute distance constraints compare to methods using similarity/dissimilarity or relative distance constraints?
\end{itemize}

\begin{figure}[t]
\begin{center}
%\missingfigure{\tiny{Two 2D plots side-by-side. Left plot showing 4 points connected to a central point. Connecting lines are labeled by Euclidean distance, points are labeled by ground-truth distance. Top and bottom point should have large distance (e.g. 3,4) while middle points should have small distance (e.g. 1), but now all have about 2 distance. Right plot shows points in transformed space, the center points are now closer to 1 distance while the other two points are now closer to 3, 4 distance. Also show metrics as ellipses (covariance matrices) making clear that our method learns these parameters.}}
\includegraphics[width=\textwidth]{Figures/approach_overview_sketch}
\caption{The effects of metric alignment on a toy problem. The left plot shows the original space, while the right plot shows the original space transformed using the learned distance function. Points are labeled with target distance, while connections are labeled by Euclidean distance in the original and transformed space.}
\label{fig:approach_overview}
\end{center}
\end{figure}
\todo{Make a plot for Figure \ref{fig:approach_overview}.}

%4.2 classification problems with continuous loss function
%4.3 kNN for structured prediction, performs best when distance function is small if loss is small for k=1, have to predict loss from feature space
%1.4 real-valued loss function instead of nominal values for classes, metric proportional to loss, predict loss from feature space by learning metric on real-valued constraints
%2.4-2 we train on real-valued ground-truth metric, such as real-valued loss function in structured prediction problems
There is a specific class of machine learning problems that requires absolute distance constraints, namely \emph{structured prediction} problems. Structured prediction problems are classification problems in which the goal is not to predict a single label, but to predict a complex output structure. For example, predicting a label sequence. Unlike a single class label, which is always either correct or incorrect, structured outputs can be partially correct. Therefore, structured prediction problems define a real-valued loss function which measures the divergence between two output structures. When we apply a prediction algorithm our goal is to predict an output that has the lowest loss relative to a given ground-truth output structure. Modifying the \ac{kNN} classification algorithm to work as structured prediction algorithm is straightforward: instead of a simple voting scheme, the outputs corresponding to the $k$ nearest neighbors in feature space are combined in some task-specific way. Since our goal is to predict output structures that have the lowest loss relative to the ground truth, this method is most effective when small distances in feature space correspond to low loss. By generating absolute distance constraints which constrain distances between points in feature space to be proportional to the loss between the corresponding outputs, we can learn a distance function that ensures nearest neighbors in feature space correspond to low-loss solutions. We refer to this method as \emph{metric alignment} since satisfying these constraints aligns the feature space to the structured-output space in which the loss function is measured.

Figure \ref{fig:approach_overview} shows an example of the effect of learning a distance function using absolute distance constraints on a small toy problem. The plot on the left shows a set of training points in the original feature space. Each point is labeled with a target distance to the unlabeled point in the center. The connecting lines between the points and the center point are labeled with the Euclidean distance in the feature space. As can be seen, the target distances and the actual distances do not match up: the upper and lower points are too close to the center, while the ones in the middle are too far away. We use the target distances in our constraints to learn a distance function and use this function to transform the original space. The result is that the distances in the transformed feature space now align much more closely to the target distances.

%contributions: describes method for metric learning with real-valued constraints, describes evaluation problem and datasets, evaluates described and related methods on real-valued constraints
This thesis describes the metric alignment method in detail and compares it against existing methods for metric learning. We apply our method to images taken from a semantic segmentation problem in order to predict semantic overlap between image patches, and to images from an attribute-based classification problem in order to predict an attribute vector from the image features. We compare our method against existing methods where we threshold the real-valued loss in order to generate similarity/dissimilarity constraints. In summary, the main contributions of this thesis are as follows:
\begin{itemize}
\item We describe a method for metric learning with absolute distance constraints in the context of structured prediction problems.
\item We introduce two learning problems based on existing datasets that can be used to evaluate metric learning methods with absolute distance constraints.
\item We evaluate our method against existing methods on these learning problems.
\end{itemize}



%%%%%%%%%%%%%%%%%%%%%%%%%%%%%%%%%%%%%%%%%%%%%%%%%%%%%%%%%%%%%%%%%%%%%
%
% BACKGROUND
%%%%%%%%%%%%%%%%%%%%%%%%%%%%%%%%%%%%%%%%%%%%%%%%%%%%%%%%%%%%%%%%%%%%%

\pagebreak
\section{Background}


\subsection{Unsupervised metric learning}


\subsection{Supervised metric learning}
\label{sec:metric_learning}



Many machine learning methods depend on pairwise comparisons between object representations. 

Euclidean distance does not always correspond with semantic distance or similarity.

The aim of metric learning methods is to learn a distance function helpful to the problem.

%local vs global, supervised vs unsupervised, linear vs non-linear



\subsubsection{Notation and problem formulation}


The problem of metric learning is to learn the parameters for a distance function from training examples.

A metric or distance function is a function $d: \mathcal{X} \times \mathcal{X} \rightarrow \mathbb{R}^{+}$, having metric or semi-metric properties (look these up).

The Mahalanobis metric is a linear parametric distance function and is most commonly used in metric learning.

% Linear metrics can be interpreted as a linear transformation on the input space.

% Although some methods learn 

structured prediction, predict complex outputs, sequences or trees, combinatorially many possible outputs, 0-1 loss not applicable, metric 

%need to parameterize the distance function, need some form of training data
%parametric distance metric: Mahalanobis distance, changing the matrix $M$ changes the distance metric, we can learn $M$ such that the distance satisfies our constraints
%Suppose we have a pair of vectors $\vec{x}, \vec{y} \in \mathbb{R}^D$
%\begin{equation}
%d_{M}(\vec{x},\vec{y}) = \sqrt{(\vec{x}-\vec{y})^{T} \mat{M} (\vec{x}-\vec{y})}
%\label{eq:mahalanobis_distance}
%\end{equation}
%metric: identity of indiscernibles: d(x,y) = 0 => x = y, symmetric: d(x,y) = d(y,x), non-negative: d(x,y) >= 0, sub-additive (triangle inequality): d(x,y) + d(y,z) >= d(x,z)
%pseudo: relaxes identity of indiscernibles to: d(x,x) = 0, meaning that there might be y not equal to x such that d(x,y) = 0
%Mahalanobis distance is pseudo-metric if $M$ is positive semi-definite and metric if $M$ is positive definite
%Linear metric, can be rewritten as linear transformation of the input space, comes in handy for nearest neighbor retrieval since we can still use Euclidean distance between transformed points
%rewrite $M = L^T L$, learn L, M is always PSD and the metric thus a pseudo metric, L is a linear transformation of the feature space such that Euclidean norm of transformed difference vector (or difference between transformed vectors) corresponds to $d_M$
%\begin{equation}
%d_{M}(\vec{x},\vec{y}) = ||\mat{L}(\vec{x}_{i}-\vec{y}_{i})||
%\label{eq:mahalanobis_transformation}
%\end{equation}
%$||\mat{L}(\vec{x}_{i}-\vec{y}_{i})|| = ||\mat{L}\vec{x}_{i}-\mat{L}\vec{y}_{i}||$


\subsubsection{Learning from equivalence constraints}

Suppose we have a problem with discrete labels for the examples, for classification or clustering we would like distances between examples to be small if their labels are the same and large otherwise.

\todo{Describe these methods (related work)}

%suppose we have a problem with input space $\mathcal{X} \subseteq \mathbb{R}^n$ and as output discrete class labels $\mathcal{Y} = {1, 2, ..., C}$. Then we want distance to be small iff labels are the same. kNN example, clustering example
%training set: subsets or pairs of objects which are similar. Many methods also use dissimilarity information: subsets or pairs of objects which are dissimiilar.
%try to make within-class variance small, difference between different classes large
%very applicable for classification tasks, but not so much for non-binary task such as those with a structured loss function, have to select some arbitrary cut-off for similarity or dissimilarity

The \acf{ITML} method \cite{davis2007information} considers similarity constraints of the form $d_{\mat{M}}(\vec{x}_{i}, \vec{x}_{j}) \leq u$ and dissimilarity constraints of the form $d_{\mat{M}}(\vec{x}_{i}, \vec{x}_{k}) \geq \ell$, where $u$ and $\ell$ are numeric upper and lower bounds on the distance between similar points pairs $\vec{x}_{i}, \vec{x}_{j}$ and dissimilar point pairs $\vec{x}_{i}, \vec{x}_{k}$. A matrix $\mat{M}$ is learned that satisfies these constraints while minimizing LogDet divergence between an input matrix $\mat{M}_0$ and $\mat{M}$.


$d_{\mat{M}}(\vec{x}_{i}, \vec{x}_{j}) + 1 \leq d_{\mat{M}}(\vec{x}_{i}, \vec{x}_{k})$
\subsubsection{Learning from ranking constraints}

Suppose we have a problem without discrete labels but in which we try to find the most similar examples, we would like the distance to a similar example to be smaller than the distance to a less similar example.

$d_{\mat{M}}(\vec{x}_{i}, \vec{x}_{j}) < d_{\mat{M}}(\vec{x}_{i}, \vec{x}_{k})$

\subsubsection{Learning from distance constraints}

Metric learning method generally don't do this, however MDS and visualization methods seem related in that they learn from pairwise distance matrix

\subsection{Nearest neighbor classification}

\subsection{Semantic segmentation}

%training set: ranking or triples of objects for which one is closer to the other
%try to satisfy as many of these constraints as possible


\subsection{SGD}
 \cite{rakhlin2012making} ASGD with suffix averaging is O(1/T) \cite{bottou2010large} ASGD or SGDQN is good for large scale problems \cite{bottou2008tradeoffs} SGD algorithms work surprisingly well in large scale learning \cite{xu2011towards} ASGD with specific parameters is optimal







%%%%%%%%%%%%%%%%%%%%%%%%%%%%%%%%%%%%%%%%%%%%%%%%%%%%%%%%%%%%%%%%%%%%%
%
% METHOD
%%%%%%%%%%%%%%%%%%%%%%%%%%%%%%%%%%%%%%%%%%%%%%%%%%%%%%%%%%%%%%%%%%%%%

\pagebreak
\section{Metric alignment}
\label{metric_regression}

%What is the problem with other metric learning methods?
%input to metric learning methods are equivalence or ranking constraints, qualitative similarity function that establish relation between pairs or triples of points, metric learning does not take advantage of continuous distance measure, discrete metric learning, have to set arbitrary threshold, ranking learning, easier to learn from a restricted sample if you have more specific information, have to generate a lot of training data, mostly just interested in brining similar points together, don't have any specific target distance function, what if you do have samples from target distance, how can you leverage this?

In existing metric learning methods the input is a set of equivalence constraints or ranking constraints. Equivalence and ranking constraints are useful when we are interested in a qualitative similarity function, such as when points are assigned to a single class and we want the distance between same-class points to be small. But, in order to apply metric learning to problems where we are interested in a real-valued similarity function, we will have to generate equivalence or ranking constraints. Ranking constraints can be generated by sampling points and sorting them according to their similarity value, while equivalence constraints can be generated by setting a threshold on the similarity value. This approach introduces extra parameters and ignores the exact similarity values, thereby throwing away useful information.

%How does our method solve this problem? What is the idea behind our method?
%metric alignment learns from real-valued distances, sampled from target distance, idea is to make Euclidean distance in feature space predictive of target distance, assumes that target metric is also a valid pseudo metric in some target space which is a linear transformation of feature space, however if not then we can still find the best linear transformation, but stay linear to keep methods efficient, just a single transformation of the features and can still use the same methods for finding neighbors/clustering

With metric alignment we take a different approach. We do not ignore the exact values, but instead train on a set of distance constraints. Distance constraints are generated by sampling pairwise similarities and encoding them as real-valued distances. These distances are then used as target distance in a regression problem. The regression problem minimizes the difference between the target distances and the learned metric distance. Like other metric learning methods we parameterize our metric as a Mahalanobis metric:  \todo{I'll probably introduce Mahalanobis distance in section 2: "Metric learning background". Move this there.}
\begin{equation}
d_{\mat{M}}(\vec{x}_i, \vec{x}_j) = \sqrt{(\vec{x}_i-\vec{x}_j)^T \mat{M} (\vec{x}_i - \vec{x}_j)}.
\label{eq:mahalanobis}
\end{equation}
By substituting $\mat{M} = \mat{L}^T \mat{L}$ we can compute the Mahalanobis distance as Euclidean distance in a linearly transformed input space:
\begin{equation}
d_{\mat{L}}(\vec{x}_i, \vec{x}_j) =  ||\mat{L} (\vec{x}_i - \vec{x}_j)||_2.
\label{eq:mahalanobis_transformation}
\end{equation}
This allows us to apply the metric alignment method as a linear transformation on the input space in a pre-processing step, and thus to use efficient methods for approximate nearest neighbor search or clustering that have been developed for Euclidean space. %Furthermore, these linear approximations have already been proven to be powerful in metric learning for discrete classes and rankings.

%How well does it work?
%toy example set that is easy to visualize, show some images of before and after transformation, also for ranking and LMNN, or otherwise show example from real data that happens to be a clear example
\todo{Show an example of effect on a toy set, a la the squares and circles in Schultz and Joachims paper.}

\subsection{Learning from pairwise distance constraints}

%What are the details of the method?
%define target distance as some function which we can sample, minimize empirical risk over this training data, global vs local sampling, training data, pairs and real-valued distance, input as pair differences, minimize using gradient descent, assuming that it is easy to sample this is large scale learning because number of pairwise distances is exponential in number of points, we use SGD, Frobenius norm regularization, many parameters: exponential in number of features, hence regularization,

A pairwise distance constraint is a tuple $(\vec{x}_i, \vec{x}_j, d_{ij})$ consisting of a pair of points in the input space $\vec{x}_i, \vec{x}_j \in \mathcal{X}$ and their target distance $d_{ij} \in \mathbb{R}_0^+$. \todo{How to get these constraints, sampling globally vs locally.}

Given a training set $\mathcal{D}$ of pairwise distance constraints we want to learn a linear transformation $\mat{L}$ that makes distances in the transformed input space predictive of target distance. We define the loss between predicted distance and target distance as squared error, thus our empirical risk corresponds to the mean squared error over the training set:
\begin{equation}
\text{MSE}_\mathcal{D}(\mat{L}) = \frac{1}{|\mathcal{D}|}\sum_{(i,j) \in \mathcal{D}} \left ( d_{ij} - ||\mat{L} (\vec{x}_i - \vec{x}_j)||_2 \right)^2.
\label{eq:mse}
\end{equation}
We fit the parameters of the transformation matrix $\mat{L}$ to the training set by minimizing the empirical risk. Thus our output transformation is given by $\hat{\mat{L}} = \argmin_{\mat{L}} \text{MSE}_\mathcal{D}(\mat{L})$. 

\subsubsection{Optimization by stochastic gradient descent}

In order to fit the transformation matrix to the training set we use \acf{SGD}. Gradient descent is an iterative first order optimization algorithm that finds a local minimum of an error function. On each iteration it changes the parameters by a small step in the direction opposite to the gradient of the error function with respect to the parameters. Since the negative gradient is the direction of steepest descent, each step leads to a reduction in error. When the error function is defined as a sum of individual errors on examples in a training set, \ac{SGD} is a stochastic version of gradient descent where at each step the gradient is computed on the error of a single example. This does not require a summation over all examples in the training set for each update, which is beneficial when the training set is large. \todo{Highlight benefits of SGD. Discuss small-scale vs large scale learning, relevant papers of Bottou and Xu. Probably move this to background and have a SGD subsection. Should also discuss ASGD and convergence speed.}

In order to apply \ac{SGD} to our learning problem we define an error function which has the same minimum as (\ref{eq:mse}) but is easier to differentiate and optimize because of its form:
\begin{equation}
E_{\mathcal{D}}(\mat{L}) =  \sum_{(i,j) \in \mathcal{D}}  \frac{1}{4} \left ( d_{ij}^2 - ||\mat{L}(\vec{x}_i - \vec{x}_j)||_2^2 \right )^2.
\label{eq:opt_target}
\end{equation}
Here $\vec{x}_i - \vec{x}_j$ is the pairwise difference between two vectors in the input space $\mathcal{X}$ and $d_{ij}^2$ is the squared target distance for this pair of input vectors. $||\mat{L}(\vec{x}_i - \vec{x}_j)||_2^2$ is the squared Euclidean distance in the transformed space which is efficient to compute as the inner product of the transformed difference vector. The constant vector $\frac{1}{4}$ is there only for convenience in differentiation. 

Interpreting the optimization objective as $E_{\mathcal{D}} = \sum_{(i,j)\in\mathcal{D}} E_{ij}$ the corresponding gradient of the error $E_{ij}$ on a single pairwise distance constraint $(i,j) \in \mathcal{D}$ is given by:
\begin{equation}
\nabla E_{ij}(\mat{L})  = - \left ( d_{ij}^2 - ||\mat{L}(\vec{x}_i - \vec{x}_j)||_2^2 \right) \mat{L} \mat{C}_{ij},
\label{eq:opt_grad}
\end{equation}
where $\mat{C}_{ij} = (x_i - x_j)(x_i - x_j)^T$ is the outer product of the pairwise difference.

We initialize our transformation as the identity transformation $\mat{L} = \mat{I}$ and then update it at each iteration as:
\begin{equation}
\mat{L} \leftarrow \mat{L} - \eta_{t} \nabla E_{ij}(\mat{L}).
\label{eq:update}
\end{equation}
Here $t$ is the iteration number, $\nabla E_{ij}(\mat{L})$ is the gradient computed over the distance constraint for points $(i,j) \in \mathcal{D}$ and $\eta_{t}$ is the learning rate: a gain parameter which controls the step size. The learning rate should decrease and the schedule with which it decreases is important for the convergence speed of the optimization \cite{xu2011towards}. We use the parameterization proposed by \cite{xu2011towards} and calculate $\eta_{t}$ at each timestep $t$ as follows:
\begin{equation}
\eta_{t} = \frac{\eta_0}{\left(1+ a \eta_0 t \right)^c}
\label{eq:eta_update}
\end{equation}
For SGD without regularization we set $a = 1$ and we leave the parameter $c$ as a hyper parameter that needs to be tuned. Following \cite{bottou2008tradeoffs} we set $\eta_0$ by optimizing it on the first thousand points in the dataset and selecting the value that reduces loss the most.

\subsubsection{Relative error}

The squared error measure in (\ref{eq:mse}) assigns greater importance to larger pairwise differences. This might not be optimal for nearest neighbor applications since the points with smallest pairwise difference are most important for classification. \todo{Mention stress functions which have parameter for this.} In order to put equal weight on different length difference vectors, we define a squared relative error as loss function:
\begin{equation}
\text{MSRE}(\mat{L}) = \frac{1}{|\mathcal{D}|}\sum_{(i,j) \in \mathcal{D}} \left ( \frac{||\mat{L} (\vec{x}_i - \vec{x}_j)||}{d_{ij}} - 1 \right)^2.
\label{eq:msre}
\end{equation}
Analogous to (\ref{eq:opt_target}) we create an optimization objective $R_{\mathcal{D}}(\mat{L}) = \sum_{(i,j) \in \mathcal{D}} R_{ij}$ which leads to the following gradient $\nabla R_{ij}(\mat{L})$ for the squared relative error on the pairwise distance constraint $(i,j) \in \mathcal{D}$:\todo{Need to handle asymptote at $d_{ij} \rightarrow 0$.}
\begin{equation}
\nabla R_{ij}(\mat{L})  =  \left ( \frac{||\mat{L}(\vec{x}_i - \vec{x}_j)||_2^2}{d_{ij}^2} - 1 \right) \frac{1}{d_{ij}^2} \mat{L} \mat{C}_{ij},
\label{eq:relative_grad}
\end{equation}
In contrast to (\ref{eq:opt_grad}) this scales the residual error and gradient by the magnitude of the squared target distance $d_{ij}^2$. Using relative error leads to larger steps for updates computed on small distances, making nearest neighbors more important in the optimization.


\subsubsection{Regularization}

%learn on training set, distances are sampled, might contain noise, might not be entirely in line with real distribution, might contain sampling bias especially for local sampling, hence possibility for overfitting, many parameters ($D^2$ for a $D$-dimensional input space) 
The target distances in the training set need not be a perfect representation of the actual distribution of distances. The sampling process might be noisy and the samples might not be independent or identically distributed. Furthermore, the number of parameters that we fit is large: $D^2$ for a $D$-dimensional input space. This flexibility in combination with imperfect training data might lead to \emph{overfitting}. The optimization continues to decrease error on the training data, but error on unseen data (e.g. a validation set) stops decreasing and starts increasing as the learned transformation becomes more specific to the irregularities of the training data and less generally applicable.

One way to prevent overfitting is by early stopping, for example by monitoring the error on a separate validation set and stopping once this error no longer decreases. Another way that we describe here is by incorporating a regularization penalty into the objective function that we optimize. For this we define a cost function that penalizes large transformations:
\begin{equation}
C_{\mathcal{D}}(\mat{L}) =  E_{\mathcal{D}}(\mat{L}) + \lambda ||\mat{L}||_\textsc{f}^2.
\label{eq:structured_risk}
\end{equation}
Here $C_{\mathcal{D}}(\mat{L})$ is the cost function that we will minimize, $E_{\mathcal{D}}(\mat{L})$ is the error on the data, $ ||\mat{L}||_\textsc{f}^2$ signifying the Frobenius norm of $\mat{L}$ is the regularization penalty and $\lambda \in \mathbb{R}_0^+$ is a parameter that determines the strength of the regularization. The gradient for the update then becomes:
\begin{equation}
\nabla C_{ij}(\mat{L}) =  E_{ij}(\mat{L}) + \lambda 2\mat{L}.
\label{eq:structured_grad}
\end{equation}


\subsubsection{Mini-batch optimization}

\todo{This should also be in the SGD background section, making it just a parameter which we'll discuss further in the tuning section.}

\begin{equation}
\mat{L} \leftarrow \mat{L} - \eta_{t} \frac{1}{|\mathcal{B}_t|} \sum_{(i,j) \in \mathcal{B}_t} \nabla E_{ij}(\mat{L})
\label{eq:update_batch}
\end{equation}





%suppose a similarity measure which gives a real valued measure of similarity between two objects, inverted (dissimilarity measure) gives us a distance function or metric that we want to learn,
%uses training data in a way that makes the method applicable to problems characterized by real-valued distance functions, such as a structured loss function
%Applicable to problems characterized by real-valued distance functions, such as a structured loss function.
%We want distance to be inverse proportional to similarity. 
%Large margin approaches lead to dissimilar examples having sufficiently larger distance than similar examples, but does not enforce the more dissimilar examples to be further away than slightly dissimilar examples proportionally to their difference in dissimilarity. 
%Ranking constraint approaches do enforce this ranking, but doesn't enforce the proportionality.
%Metric alignment directly approximates given real-valued dissimilarity measure and therefore satisfies all these constraints.
%we would like examples with a larger loss to have greater distance.

%Metric alignment is posed as a simple convex optimization problem which can be solved without special purpose solvers.
%Solution can be written as an unconstrained convex optimization problem.

%training set: pairs of objects with given distance (not similarity since we learn a distance function)
%$\mathcal{X}_{\textsc{train}} \subseteq \mathcal{X}$
% training tuples $\mathcal{T} = \{(\vec{x}_i, \vec{x}_j, d_{ij})\}$ with input vectors $\vec{x}_i, \vec{x}_j \in \mathcal{X}_{\textsc{train}}$ and their pairwise distances $d_{ij} \in \mathbb{R}_{0}^{+}$.
%Suppose we have a problem in which we are interested in the similarity between input examples. These input examples are represented through a set of real-valued attributes, thus each example corresponds to a vector  $\vec{x} \in \mathbb{R}^n$ in the representation space. We now want to measure the similarity between two input vectors $\vec{x}_i$ and $\vec{x}_j$. Without any prior knowledge we could measure the Euclidean distance between 
% compare $d(\vec{x}_i, \vec{x}_j) = ||\vec{x}_i - \vec{x}_j||_2$
%learn a distance metric specific to the problem instead of relying on 'default' metrics (e.g. Euclidean), same data set and features but difffernt task
%learn M by minimizing SSE over this training set
%\begin{equation}
%M = \argmin_{M} \sum_{(\vec{x}_i, \vec{x}_j, d_{ij}) \in \mathcal{T}} \left(d_{ij} -  \sqrt{(\vec{x}_{i}-\vec{x}_{j})^{T} M (\vec{x}_{i}-\vec{x}_{j})}\right)^2
%\label{eq:min_sse_M}
%\end{equation}
%rewrite $M = L^T L$, learn L, M is always PSD and the metric thus a pseudo metric, L is a linear transformation of the feature space such that Euclidean norm of transformed difference vector (or difference between transformed vectors) corresponds to $d_M$
%\begin{equation}
%L = \argmin_{L} \sum_{(\vec{x}_i, \vec{x}_j, d_{ij}) \in \mathcal{T}} \left( d_{ij} -  ||L(\vec{x}_{i}-\vec{x}_{j})|| \right)^2
%\label{eq:min_sse_L}
%\end{equation}
%\begin{equation}
%L = \argmin_{L} \sum_{(\vec{x}_i, \vec{x}_j, d_{ij}) \in \mathcal{T}} \left( d_{ij}^2 - L\delta_{ij} \cdot L\delta_{ij} \right)^2
%\label{eq:min_sse_L}
%\end{equation}

%%%%%%%%%%%%%%%%%%%%%%%%%%%%%%%%%%%%%%%%%%%%%%%%%%%%%%%%%%%%%%%%%%%%%
%
% EXPERIMENTS
%%%%%%%%%%%%%%%%%%%%%%%%%%%%%%%%%%%%%%%%%%%%%%%%%%%%%%%%%%%%%%%%%%%%%


\section{Results}

To evaluate metric alignment we need a dataset for which a real-valued target metric is defined. We took our testing data from a semantic segmentation benchmark. Semantic segmentation is a computer vision problem in which the objective is to segment input images into contiguous regions by assigning labels to each pixel from a predefined set of class labels. 


Segmentation benchmarks consist of images $\mat{I}$ where each pixel $\mat{I}_{(i,j)}$ is usually given as a triple $(r,g,b)$ denoting the color of a single pixel and segmentations $\mat{S}$ of which each element  $\mat{S}_{(i,j)} \in \mathcal{L}$ assigns a label to pixel $\mat{I}_{(i,j)}$ of the image. Here $\mathcal{L} = \{1, 2, \ldots, n\}$ is a pre-determined label set  where each label corresponds to one of $n$ visual classes. For our dataset we extract square patches from the images and the corresponding segmentations.


\begin{equation}
L(\vec{y}, \vec{y}^\prime) = 1 - \frac{\sum_{i,j} \left [ y_{i,j} = y^\prime_{i,j}\right]}{\sum_{i,j} \left [y_{i,j}, y^\prime_{i,j} \in \mathcal{L}\right ]}.
\label{eq:patch_loss}
\end{equation}



\section{Application to semantic segmentation}




%semantic segmentation, per-pixel classification, apply \ac{kNN} to image patches, patch has feature descriptor and label patch, labels correspond to object classes, loss is distance in label space, align feature space to label space, improved segmentation accuracy

%semantic segmentation, challenging problem, combines object recognition and image segmentation, applications in image understanding, image retrieval
Semantic segmentation is a challenging problem that combines object recognition and image segmentation, with applications in image understanding and image retrieval. The objective is to segment input images into contiguous regions by assigning labels to each pixel from a predefined set of class labels, in effect performing a per-pixel classification of the image. Figure \ref{fig:segmentation_examples} shows some example input and output for the semantic segmentation task and we give a more formal description below.

\begin{quotation}
\subsubsection*{Semantic segmentation problem.}

\paragraph{Input:} An image $\mat{I}$ where each pixel $\mat{I}_{(i,j)}$ is usually given as a triple $(r,g,b)$ denoting the color of a single pixel. And a label set $\mathcal{L} = \{1, 2, \ldots, n\}$ where each label corresponds to one of $n$ pre-determined object classes.

\paragraph{Output:} A segmentation image $\mat{S}$ of which each element  $\mat{S}_{(i,j)} \in \mathcal{L}$ assigns a label to pixel $\mat{I}_{(i,j)}$ of the image.

\paragraph{Learning problem:} given a set of training images $\mathcal{I}_\textsc{train} \subset \mathcal{I}$ with the corresponding ground truth segmentations $\mathcal{S}_\textsc{train} \subset \mathcal{S}$ and a loss function $L: \mathcal{S} \times \mathcal{S} \rightarrow \mathbb{R}$, learn a function $h(\mat{I}) = \mat{S}$ that minimizes this loss over unseen images. %\todo{...minimizes this loss over true distribution of images.}
\end{quotation}


Because the output consists of a 2-dimensional label array this is a structured prediction problem. We solve this problem by combining \ac{kNN} classification with metric alignment. First we divide images into overlapping patches and compute a feature vector for each patch. We use metric alignment to learn a transformation of these feature vectors that aligns distances in feature space with loss in label space. Then we classify each patch using structured \ac{kNN} classification.

\begin{figure}[tbh]
\begin{center}
\missingfigure{Top row: image from MSRC, VOC, CamVid. Bottom row: corresponding segmentations.}
\caption{Some examples of the semantic segmentation task. Top row: input images from the MSRC, Pascal VOC, and CamVid benchmarks. Bottom row: the manual segmentations corresponding to the input images. Each color denotes a different object class. Note the difference in annotation style between the benchmarks.}
\label{fig:segmentation_examples}
\end{center}
\end{figure}

% common approaches, structured learning CRF, tree ensembles with context features, segmentation with object detectors, clustering
%The main challenges in semantic segmentation are a combination of the challenges found in object recognition, image segmentation, and structured prediction. Like in object recognition there is high variance in appearance, orientation and shape of objects belonging to object classes of interest. However, since single pixels have to be labeled, there is less information available for making a labeling decision. Semantic segmentation methods solve this either by pre-segmenting the image into superpixels or larger segment hypotheses such that evidence, e.g. object recognition response, can be accumulated over a larger area, or by using features which aggregate information about a larger area around the pixel, its context. Like in structured prediction problems, maintaining consistency between predicted labels is also an important challenge. Because of this, many semantic segmentation methods use a \ac{HCRF} approach which allows the penalization of inconsistency on multiple levels unless there is strong evidence for non-continuity in the labeling.\todo{Give some global background describing challenges of segmentation (much variance in appearance and shape, label consistency, finding correct scales for segmentation) and the main approaches to solving them (HCRF, superpixels, SVM, which differ by benchmark).}\todo{Segmentation box? Show different segmentation tasks and talk about basic methods like superpixels and clustering methods?}

\subsection{Structured \acl{kNN} segmentation}

%segmentation with kNN, kNN is simple, requires little parameter tuning, and is applicable to multi-class problems, 
Our segmentation method is based on \acf{kNN} classification. \ac{kNN} is a non-parametric classification method, which means that its decision function is not characterized by a parameterized model. Instead, the class for an input point is decided to be the majority class amongst the most similar points in the training set, the nearest neighbors. Although it requires storing of the whole training set, which may become problematic for large sets, the \ac{kNN} classification method is simple, requires little parameter tuning and is inherently applicable to multi-class problems. \todo{Box: the k-nearest neighbor classification?}

%we use a simple generalization to structured prediction, this is patch based, find nearest patches and interpolate each label of the patch independently
We generalize \ac{kNN} to the case of structured prediction by applying it to structured label patches. %, following the idea in \cite{kontschieder2011structured} of generalizing random forests to use structured label patches. 
We define a structured output space $\mathcal{Y} = \mathcal{L}^{d \times d}$ consisting of label patches $\vec{y}$ whose elements $y_{ij} \in \mathcal{L}$ each consist of one of the pre-determined class labels. When predicting the label patch $\hat{\vec{y}}$ for a given input vector we look up the label patches $\mathcal{Y}_\textsc{nn}$ of the $k$ nearest neighbor vectors  and interpolate each label position independently based on the labels at the same position $(i,j)$ of the nearest neighbor patches:
\begin{equation}
\hat{y}_{ij} = \argmax_{\ell \in \mathcal{L}} \sum_{\vec{y} \in \mathcal{Y}_\textsc{nn}} \left[ y_{ij} = \ell \right ].
\label{eq:nn_decision_rule}
\end{equation}
Here $\hat{y}_{ij}$ is the pixel at position $(i,j)$ of the output patch $\hat{\vec{y}}$ and $\left [\ \cdot \ \right ]$ denotes the Iverson bracket which is equal to 1 if the enclosed statement is true and 0 otherwise. This computation corresponds to letting each pixel of each neighbor patch vote on the class of the corresponding output pixel and then taking the most voted class for that output pixel.

\begin{figure}[tbh]
\begin{center}
\missingfigure{Figure explaining searching for nearest neighbor vectors in feature space and patch creation from neighbor patches according to equation \ref{eq:nn_decision_rule}}
\caption{Schematic overview of the \ac{kNN} algorithm with structured label patches. }
\label{fig:knn_segmentation}
\end{center}
\end{figure}

%for each of the training images extract $d \times d$ image patches from $I$ on regular grid $(\vec{u}, \vec{v})$, compute array of image features $X = \left [ \vec{x}_1, \vec{x}_2, \ldots, \vec{x}_p \right ]$ with $\vec{x}_i$ computed from the image patch centered on image coordinates $(u_i, v_i)$, use SIFT and CSIFT features, also extract label patches from the corresponding image labeling $L$ such that label patch  $\vec{y}_i$ corresponds to the $d' \times d'$ label patch around $(u_i, v_i)$, store these coordinates and create index for the features such that we can efficiently find the closest points
From each training image $\mat{I}$ we extract feature vectors $\mathcal{X}_\mat{I} = \left\{ \vec{x}_1, \vec{x}_2, \ldots, \vec{x}_p \right\}$ on a regular hexagonal grid $(\vec{u}, \vec{v})$, with $\vec{x}_i$ computed from the $d^\prime \times d^\prime$ image area centered on image coordinates $(u_i, v_i)$. Each of these feature vectors $\vec{x}_i$ is paired to a label patch $\vec{y}_i$ of size $d \times d$ taken from the training annotation for the image and centered at $(u_i, v_i)$. Note that the size $d$ for the label patch and $d^\prime$ for the feature patch can be different. Specifically, $d^\prime$ is dependent on the scale of the image features while $d$ is a hyper-parameter that we set. For convenience we define these patches to have the origin $y_{0,0}$ in the center of the patch, such that the subscripts $i$ and $j$ run from $-\tfrac{d}{2}$ to $\tfrac{d}{2}$. Together these feature vectors form a training set $\mathcal{X}_\textsc{train} = \bigcup\left\{\mathcal{X}_\mat{I}: \mat{I} \in \mathcal{I}_\textsc{train} \right\}$ We store all the training vectors and label patches and we create a randomized kd-trees index for the feature vectors using FLANN \cite{muja2009fast} such that we can efficiently find the closest points from this set given an input vector.

%for each test image, again compute features for each patch, for each feature find the closest patch in the training set, extract label patch from the same area, merge overlapping label patches to create the output labeling
For each test image $\mat{I}$ we again extract feature vectors $\mathcal{X}_\mat{I}$ on the grid $(\vec{u}, \vec{v})$. For each of these feature vectors we search for the nearest neighbors amongst the training vectors and their associated label patches. From these we then generate the hypothesis patches as per (\ref{eq:nn_decision_rule}) giving us a label patch centered at each grid point $(u_i, v_i)$. These hypothesis patches are partially overlapping. Thus for each pixel in the image we select the most voted class among the positions in patches that overlap that pixel, similar to (\ref{eq:nn_decision_rule}).

%\begin{algorithm}
%\caption{The structured k-nearest neighbor algorithm for label patches.}
%\label{alg:structured_kNN}
%\begin{algorithmic}[1]
%\REQUIRE  $\mat{X}_\textsc{train}$, $\mat{Y}_\textsc{train}$, $\mat{X}_\mat{I}$ 
%\ENSURE $S = h(\mat{I})$
%\STATE TODO: give clear pseudocode
%\end{algorithmic}
%\end{algorithm}

\subsection{Aligning distances in feature space to loss}

%effectiveness of \ac{kNN} method highly dependent on representation function, meaning close in feature space has to align with semantically close, if appearance and semantics are not aligned this will lead to wrong prediction based on nearest neighbors
The effectiveness of the structured \ac{kNN} method described above is dependent on both the representation function and the distance function. The  distance function defines the way in which distances are measured in feature space, while the representation function defines the feature space. In order to be effective, our method needs semantically similar points to lay close together in feature space, with `semantically similar' meaning that the corresponding label patches have a large overlap.

%use metric alignment to learn feature transform, label patches, loss defines distance in label space, we want small distance to mean small loss
The naive approach is to use Euclidean distance as a default and then to search for features which make similar patches lie close together. However, it is very difficult to engineer such an image feature. Instead, we will use existing features which have been applied in many image classification tasks and use the metric alignment method described in section \ref{metric_regression} to learn a distance metric which makes distance between feature vectors predictive of loss between the corresponding labels. We define this loss as a continuous function $L: \mathcal{Y} \times \mathcal{Y} \rightarrow \mathbb{R}$ with range $[0,1]$ which gives us an inverse of the overlap between label patches:
\begin{equation}
L(\vec{y}, \vec{y}^\prime) = 1 - \frac{\sum_{i,j} \left [ y_{i,j} = y^\prime_{i,j}\right]}{\sum_{i,j} \left [y_{i,j}, y^\prime_{i,j} \in \mathcal{L}\right ]}.
\label{eq:patch_loss}
\end{equation}
Here $L(\vec{y}, \vec{y}^\prime)$ is the loss for predicting patch $\vec{y}^\prime$ for patch $\vec{y}$. This loss counts the positions that have been assigned the same label in both patches as a fraction of the positions that been assigned a label in both patches, since some positions are not assigned a label during annotation. Since the overlap is a similarity measure, we turn it into a distance measure by subtracting it from 1. By training the distance metric on this pairwise loss, we try to align the feature space to the label space in such a way that when we search for nearest neighbors in feature space we will find neighbors that have label patches that are close to the correct label patch for the input feature vector.


%%%%
%%%%
%%%%


\subsection{Experimental setup}

%We evaluate our method by applying our segmentation pipeline to benchmark problems with Euclidean distance and with distance metric learned by our metric alignment method.
%learning to rank, LMNN, RF (Kontschieder et al.)
To evaluate our method we perform an experiment using two semantic segmentation benchmarks. In this experiment we apply the \ac{kNN} segmentation method described above using a distance function learned by applying metric alignment on the training data. We compare this against the result of applying the \ac{kNN} method using standard Euclidean distance.

%We compare our method against two baseline large-margin metric learning methods: LMNN \cite{weinberger2009distance} which learns from discrete examples and \cite{schultz2003learning}
Besides the Euclidean baseline we also compare against two alternative metric learning methods, both of which use a large margin approach to metric learning. First against \ac{LMNN} \cite{weinberger2009distance}, described in section \ref{discrete_metrics}, and secondly against \cite{schultz2003learning}, described in section \ref{ranking_metrics}. We use \ac{LMNN} as a baseline because it is a well tested method designed specifically to improve \ac{kNN} classification results. The second method was chosen because it is similar in spirit to our own approach in that it learns from relative comparisons, but it is different in practice since it uses qualitative training data while metric alignment uses quantitative data.

%We evaluate both the accuracy of the learned metrics on pairwise differences and the effect of the metric learning on the segmentation task using our \ac{kNN} segmentation method.
To compare the different methods in this experiment we look at how well the distance function between image patches approximates the label distance as measured by \ac{MSE} as well as the overall increase in segmentation accuracy as measured by the benchmark evaluation measures.



\subsubsection{Datasets}

%We evaluate our method on the MSRC and VOC segmentation datasets; two benchmarks in semantic segmentation.
To evaluate our methods we use two datasets of images which have been manually segmented into  \footnote{\url{http://research.microsoft.com/en-us/projects/objectclassrecognition/}}

The MSRC set consists of a small number of images, loosely segmented into 21 classes. covering most of each image with little variation in appearance and scene.

Camvid dataset

The VOC set consists of a large number of images with strict segmentation into 20 foreground classes and an aggregate background class with much variation in appearance. %Object detection challenge, hence background class, but this is challenging for segmentation methods because you have to decide what not to segment.

Difference in annotation styles, MSRC is loosely segmented while the Camvid and VOC are close to pixel-perfect. VOC generally has one or a few foreground objects segmented while the rest is grouped as background, while MSRC and CamVid are fully segmented with background classes being labeled, e.g. `grass', `street'. See figure \ref{segmentation_examples}.

VOC is more challenging than MSRC and CamVid on both effectiveness and sheer size.


\subsubsection{Implementation}

We use SIFT and C-SIFT image descriptors as feature vectors extracted using the ColorDescriptor software. \cite{sande2011empowering} We then index these training vectors using the multiple kd-tree method from the FLANN library \cite{muja2009fast}, which allows us to efficiently calculate nearest neighbors amongst these points given an input vector.

\todo{Box: image descriptors?}

\subsubsection{Training and parameter selection}

We divide the dataset into non-overlapping training and testing images. 


extract training pairs from training images, note: can't use all patch pairs because this are intractably many, we use a set of randomly selected points and their k-nearest neighbors

train on feature vectors and pairwise loss, compute gradient over the neighborhood set at each step

\todo{Explain tuning.}






\subsubsection{Evaluation}



We measure the \ac{MSE} between the label distance and the learned nearest-neighbor on a sample of testing pairs

%nearest neighbor MSE histogram?

we evaluate global pixel accuracy and average class pixel accuracy as well as the intersection over union measure used in the pascal VOC dataset

we show per patch results for different numbers of neighbors 

plot of label distance vs learned metric distance, both normalized\

plot of segmentation accuracy as a function of neighbors?

plot of distance to closest ground-truth patch?





%%%%
%%%%	
%%%%








\subsection{Results}

%Training curve with MSE on training pairs

%MSE on patch overlap for testing pairs

%effect on segmentation score

%effect of features and dimension reduction on learnability

%effect of parameters for minimization, not all training schemes seem to work





We show that learning a simple linear transformation improves alignment between pairwise Euclidean distance in feature space and pairwise loss, which improves overall segmentation accuracy.




































%%%%%%%%%%%%%%%%%%%%%%%%%%%%%%%%%%%%%%%%%%%%%%%%%%%%%%%%%%%%%%%%%%%%%
%
% CONCLUSIONS
%%%%%%%%%%%%%%%%%%%%%%%%%%%%%%%%%%%%%%%%%%%%%%%%%%%%%%%%%%%%%%%%%%%%%

\pagebreak
\section{Conclusions}

%list the specific cases/requirements in which this is useful (see intro)















\bibliographystyle{plain}
\bibliography{Thesis}


%\appendix
%\section{Glossary}

%\begin{description}
%\item[Similarity measure] A measurement of the similarity between two objects. 
%\item[Distance function] A function which maps pairs of points in some space to a positive real number or zero.
%\item[Metric] See \emph{distance function}.
%\item[Metric transformation] A linear transformation of a feature space for which it holds that Euclidean distances in the transformed space correspond to a specified distance function on the original space. 
%\end{description}

\end{document}  