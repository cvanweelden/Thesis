\documentclass[a4paper,titlepage]{article}
\usepackage{listings}
\usepackage{graphicx}
\usepackage[font=small,labelfont=bf]{caption}
\usepackage{subcaption}
%\usepackage[scale=.65]{geometry}
%\usepackage{fullpage}
%\usepackage{a4wide}
\usepackage{amsmath}
\usepackage{mathtools}
\usepackage{float}
\usepackage{acronym}
\usepackage{cite}
\usepackage{url}
\usepackage[usenames, pdftex]{color}
\usepackage{soul}
\usepackage{rotating}
\usepackage{booktabs}
\usepackage{todonotes}
\usepackage{amsfonts}
\usepackage{algorithmic}
\usepackage{algorithm}
\usepackage{wrapfig}
\usepackage{lipsum}

%\newenvironment{textbox}{\medskip \begin{mdframed}}{\end{mdframed} \medskip}

\floatstyle{ruled}
\newfloat{textbox}{thp}{lob}
\floatname{textbox}{Text Box}


\renewcommand{\algorithmicrequire}{\textbf{Input:}}
\renewcommand{\algorithmicensure}{\textbf{Output:}}

\setlength{\marginparwidth}{4cm} %else todo notes will only fill halve the margin

\newcommand{\note}[1]{\marginpar{\colorbox{yellow}{\parbox{3.7cm}{#1}}}}

\renewcommand{\vec}[1]{\mathbf{#1}}
\newcommand{\mat}[1]{\mathbf{#1}}
\DeclareMathOperator*{\argmax}{arg\,max}
\DeclareMathOperator*{\argmin}{arg\,min}

\newcommand{\tr}[0]{\text{tr}}

\acrodef{MDS}{multidimensional scaling}
\acrodef{MSE}{mean squared error}
\acrodef{LMNN}{large margin nearest neighbor}
\acrodef{kNN}[$k$NN]{$k$-nearest neighbors}
\acrodef{HCRF}{hierarchical conditional random field}
\acrodef{SGD}{stochastic gradient descent}
\acrodef{ITML}{information-theoretic metric learning}
\acrodef{PSD}{positive semi-definite}
\acrodef{SVM}{support vector machine}

%\title{Metric alignment}
%\title{Aligning metric to a distance function in an unknown space.}
%\title{Metric alignment: learning distance functions that linearly approximate arbitrary similarity measures.}
%\title{Metric alignment: linear approximation of arbitrary similarity measures.}
\title{Metric alignment: metric learning with absolute distance constraints.}
%\title{Metric alignment: metric learning using absolute distance constraints with an application to structered prediction.}
%\title{}
%\title{}


\author{Carsten van Weelden  \\ \texttt{cvanweelden@gmail.com} \\ 0518824 \and \small{Supervisor:} \\ Jan van Gemert \\ \texttt{J.C.vanGemert@uva.nl} \and University of Amsterdam\\
  The Netherlands \\ \\ In partial fulfillment etc.}
%\date{}                                           % Activate to display a given date or no date

\begin{document}
\maketitle

\begin{abstract}

\end{abstract}

\pagebreak

\tableofcontents

\pagebreak


%Boxes to write:
% ++ kNN classification
% k-means clustering
% + structured prediction problems (or)
% + loss functions used in machine learning
% image segmentation problems
% zero-shot learning / attribute based learning

%\begin{textbox}
%machine learning problems: prediction: learn some function $h: \mathcal{X} \rightarrow \mathcal{Y}$, input domain are features, most algorithms focus on simple output domains:
%\begin{itemize}
%\item $\mathcal{Y} = {-1, +1}$ binary classification
%\item $\mathcal{Y} = {1, 2, ..., n}$ classification
%\item $\mathcal{Y} = \mathbb{R}$ regression
%\end{itemize}
%
%loss functions
%
%structured prediction
%\end{textbox}




%3.5(bad) student network example
%3.7 proposed method should be used when you need a distance function measuring similarity or a feature space in which Euclidean distances correspond to similarity between objects for which it is difficult to decide whether they belong to the same discrete class and the similarity you want to measure is known for at least a subset of the training set but not for new examples but training and test objects can be represented by the same set of features.

%4.1 metric alignment, continuous distance functions, unknown or unavailable during execution, transforms feature space, feature distances are predictive of training distance
%3.4-2 metric alignment, learn a metric that satisfies real similarity score



%dimension reduction and visualization methods, (multi-dimensional scaling, isomap, LLE, SNE), mapping/embedding retaining pairwise distances, finding coordinates in some space which satisfy a similarity score



\section{Introduction.}

%2.1 objects represented as vectors of feature values, feature space
%1.1-1 machine learning features, feature space,
In machine learning tasks, objects of interest are represented using a set of features. Features correspond to attributes of the object that can be encoded in some way. For example the features representing images may simply be the pixel values of an image, but may also be more complex features computed from the basic pixel values. Each object is thus represented by a feature vector containing the feature values. We will refer to the space of all possible feature vectors as the \emph{feature space} and in the case of $d$-dimensional real-valued vectors take this space to be $\mathbb{R}^d$. \todo{Too abstract, might loose some readers in the first paragraph. Too fast into the mathematical notation. Maybe start with application. Show an image of the example (maybe replace cars with the animals set). Try to also show ground truth metric in an image.}

%2.2-1 measure similarity by assuming small difference in feature vector means similar objects,
%1.1-2 measuring distance, finding similar samples, Euclidean distance
%computing distance between data samples fundamental problem
%3.1-1 similarity between objects, distance function,
Now suppose we would like to measure similarities between objects. For example, we have some images of a Ferrari car and we want to find images of similar cars. Assuming that similar objects result in similar feature values, we can express similarity between objects as a distance function in feature space. For our $\mathbb{R}^d$ feature space we can therefore measure similarity by Euclidean distance between the points in the feature space given by the feature vectors. Thus, starting from the feature representations of our input images we would expect to find images of similar cars nearer in feature space than images of dissimilar cars.

%3.2 representation of objects, feature space, distance in this space different from similarity, explain problem with clustering: picking different features or scaling will lead to different clusters which do not all correspond to our idea of similarity, common to use Euclidean or Hamming distance depending on the feature vector
%2.3-1 feature difference not always good measure for similarity,
However, Euclidean distance is not necessarily the best measure for similarity. In most cases features are predetermined since they measure basic attributes of the object, but the type of similarity we want to measure is specific to the task. With Euclidean distance each feature has an equal impact on the distance, even though some features are much more relevant to the type of similarity we want to measure than others. For example in our car images example, we would want to focus on similarities in color if we are simply looking for other red cars, but we would want to look at more complex similarities in shape if we are looking for other racing cars. 

%1.2 use of metric in machine learning methods, kNN, k-means, SVM, effectiveness depends on metric
%2.2-2 used by kNN and k-means
%3.1-2 applications (clustering, non-parametric methods based on distance or local density [kernel density estimation, k-nearest neighbor classification], visual identification)
Several machine learning methods use a distance function at the core of their algorithm. For example, the \ac{kNN} algorithm classifies objects by searching the training data for feature vectors that are closest to the input vector. The objects corresponding to these nearest neighbors then vote on the class to assign to the input object. It is clear that for this to be effective, distances in feature space should be small for objects of the same class and large for objects of differing class. Similarly, the $k$-means clustering algorithm clusters samples based on their distances in feature space. This will cluster similar objects together if they are close together in feature space. These methods require us to carefully choose a feature representation and distance function in order to be effective.

%1.3-1 metric learning, parametric distance function, learn parameters,
%2.3-2 metric learning: learn distance function that is more useful
%3.3-1 metric learning: learn a distance function between object representations that corresponds to actual similarity, metric, parametric metrics, metric transform, Mahalanobis distance,
\todo{Could sell this as: Since we have some background knowledge about what we want the distance function to be like, we can use this knowledge to automatically learn ...}
Instead of hand-crafting a distance function, we can automatically learn a suitable distance function in a given feature space. This approach is called \emph{metric learning}. Metric learning methods define a parametric distance function and then learn the best parameters for this function. Learning the parameters requires a set of training samples in the given feature space and a set of constraints on the distances between training samples. Parameters of the distance function are learned that best fit these distance constraints. The constraints thus represent the similarity measure specific to our task. For example, if we are still looking for images of similar racing cars, we would provide samples of racing cars and then constrain the distances between these to be small, while constraining the distances to other types of cars to be large.

%1.3-2 constraints based on class of samples
%2.4-1 train metric on class differences,
%3.3-2 using discrete similarity for training or a similarity ranking,
Metric learning has been researched most in the context of multi-class classification problems and ranking problems. Because of this background, the constraints generally take the form of similarity/dissimilarity constraints or relative distance constraints. Similarity and dissimilarity constraints are generated from class labels: objects of the same class are constrained to have a small pairwise distance while objects of different classes are constrained to have a larger distance. In contrast, relative distance constraints are used for tasks where it might be difficult to make absolute similarity judgments, but where we can identify one object to be more similar to a given object than some other object. Relative distance constraints constrain one pairwise distance to be smaller than another pairwise distance.

%3.4-1 what if we have exact similarity scores represented as real numbers, but not discrete classes, e.g. structured problem
%research questions: how can metric learning be applied to real-valued loss, how does this compare to existing problems
However, there is a more direct approach to the metric learning problem: instead of generating similarity/dissimilarity or relative distance constraints, a metric can be trained on absolute distance constraints that directly specify what the distance between a given pair of points in feature space should be. \todo{Put more emphasis on target metric. Next to having labels we also have an exact idea of what the metric should look like. We differentiate from other methods in that we can use a target metric and align to that, while other methods can not do this.} Absolute distance constraints make more fine-grained similarity judgments than the binary similarity/dissimilarity constraints while supplying stronger supervision than relative distance constraints. Despite these advantages, absolute distance constraints have received little attention in the metric learning literature. Therefore, this thesis investigates the use of absolute distance constraints for metric learning. In doing so it answers the following research questions:
\begin{itemize}
\item How can we apply metric learning methods to problems with absolute distance constraints?
\item How does metric learning with absolute distance constraints compare to methods using similarity/dissimilarity or relative distance constraints?
\end{itemize}

\begin{figure}[t]
\begin{center}
%\missingfigure{\tiny{Two 2D plots side-by-side. Left plot showing 4 points connected to a central point. Connecting lines are labeled by Euclidean distance, points are labeled by ground-truth distance. Top and bottom point should have large distance (e.g. 3,4) while middle points should have small distance (e.g. 1), but now all have about 2 distance. Right plot shows points in transformed space, the center points are now closer to 1 distance while the other two points are now closer to 3, 4 distance. Also show metrics as ellipses (covariance matrices) making clear that our method learns these parameters.}}
\includegraphics[width=\textwidth]{Figures/approach_overview_sketch}
\caption{The effects of metric alignment on a toy problem. The left plot shows the original space, while the right plot shows the original space transformed using the learned distance function. Points are labeled with target distance, while connections are labeled by Euclidean distance in the original and transformed space.}
\label{fig:approach_overview}
\end{center}
\end{figure}
\todo{Make a plot for Figure \ref{fig:approach_overview}.}

%4.2 classification problems with continuous loss function
%4.3 kNN for structured prediction, performs best when distance function is small if loss is small for k=1, have to predict loss from feature space
%1.4 real-valued loss function instead of nominal values for classes, metric proportional to loss, predict loss from feature space by learning metric on real-valued constraints
%2.4-2 we train on real-valued ground-truth metric, such as real-valued loss function in structured prediction problems
There is a specific class of machine learning problems that requires absolute distance constraints, namely \emph{structured prediction} problems. Structured prediction problems are classification problems in which the goal is not to predict a single label, but to predict a complex output structure. For example, predicting a label sequence. Unlike a single class label, which is always either correct or incorrect, structured outputs can be partially correct. Therefore, structured prediction problems define a real-valued loss function which measures the divergence between two output structures. When we apply a prediction algorithm our goal is to predict an output that has the lowest loss relative to a given ground-truth output structure. Modifying the \ac{kNN} classification algorithm to work as structured prediction algorithm is straightforward: instead of a simple voting scheme, the outputs corresponding to the $k$ nearest neighbors in feature space are combined in some task-specific way. Since our goal is to predict output structures that have the lowest loss relative to the ground truth, this method is most effective when small distances in feature space correspond to low loss. By generating absolute distance constraints which constrain distances between points in feature space to be proportional to the loss between the corresponding outputs, we can learn a distance function that ensures nearest neighbors in feature space correspond to low-loss solutions. We refer to this method as \emph{metric alignment} since satisfying these constraints aligns the feature space to the structured-output space in which the loss function is measured.

Figure \ref{fig:approach_overview} shows an example of the effect of learning a distance function using absolute distance constraints on a small toy problem. The plot on the left shows a set of training points in the original feature space. Each point is labeled with a target distance to the unlabeled point in the center. The connecting lines between the points and the center point are labeled with the Euclidean distance in the feature space. As can be seen, the target distances and the actual distances do not match up: the upper and lower points are too close to the center, while the ones in the middle are too far away. We use the target distances in our constraints to learn a distance function and use this function to transform the original space. The result is that the distances in the transformed feature space now align much more closely to the target distances.

%contributions: describes method for metric learning with real-valued constraints, describes evaluation problem and datasets, evaluates described and related methods on real-valued constraints
This thesis describes the metric alignment method in detail and compares it against existing methods for metric learning. We apply our method to images taken from a semantic segmentation problem in order to predict semantic overlap between image patches, and to images from an attribute-based classification problem in order to predict an attribute vector from the image features. We compare our method against existing methods where we threshold the real-valued loss in order to generate similarity/dissimilarity constraints. In summary, the main contributions of this thesis are as follows:
\begin{itemize}
\item We describe a method for metric learning with absolute distance constraints in the context of structured prediction problems.
\item We introduce two learning problems based on existing datasets that can be used to evaluate metric learning methods with absolute distance constraints.
\item We evaluate our method against existing methods on these learning problems.
\end{itemize}



\section{Background}


%learning problems notation, \mathcal{X}, vectors in Euclidean space

%Euclidean distances

%Mahalanobis distance box

%Parametric distance

%Constraint types

%Optimization, solving in A vs in G, constrained (A PSD) vs unconstrained

%Regularization

%Fixed rank learning


Metric learning methods learn a distance function specific to a task through supervised learning. Supervised learning requires a training set containing data points and some sort of side information about those points which informs the learning process. In practice, this side information comes in the form of class labels for each point in classification problems or the correct structured output in structured prediction problems. This side information is often referred to as the ground-truth.

Training points are assumed to be $d$-dimensional real-valued feature vectors in a Euclidean feature space $\mathcal{X} = \mathbb{R}^d$. Let $\mat{X} = \left[ \vec{x}_1, \vec{x}_2, ..., \vec{x}_n \right]$ be the matrix of all the training points, with each column being a single feature vector $\vec{x}_i \in \mathcal{X}$. Each element of a training vector corresponds to a single feature value. Distances between points in the feature space can be measured by the Euclidean distance $\|\vec{x}_i - \vec{x}_j\| = \sqrt{(\vec{x}_i - \vec{x}_j)^T(\vec{x}_i - \vec{x}_j)}$, although in practice the squared distance is often used since dropping the square root makes the distance easier to compute.



\subsection{Linear distance functions}

The goal of metric learning is to learn a distance function in the feature space. A distance function, or metric, is a function $d: \mathcal{X} \times \mathcal{X} \rightarrow \mathbb{R}^{+}$ having the following properties:
\begin{subequations}
\begin{align}
        d(\vec{x}_i, \vec{x}_j) & \geq 0 &  &\text{non-negativity}, \label{eq:non-negativity}\\
        d(\vec{x}_i, \vec{x}_j) & = 0 \leftrightarrow i = j & &\text{identity of indiscernibles}, \label{eq:identity_of_indiscernibles} \\
        d(\vec{x}_i, \vec{x}_j) & = d(\vec{x}_j, \vec{x}_i) &  &\text{symmetry}, \label{eq:symmetry}\\
        d(\vec{x}_i, \vec{x}_k) & \leq d(\vec{x}_i, \vec{x}_j) + d(\vec{x}_j, \vec{x}_k) &  &\text{triangle inequality}. \label{eq:triangle_unequality}
\end{align}
\end{subequations}
In addition, a distance function is called a pseudo-metric if it conforms to all these conditions except (\ref{eq:identity_of_indiscernibles}) which is replaced by $d(\vec{x}_i, \vec{x}_i) = 0$, meaning that points still have zero distance to themselves but might also have zero distance to other points. Metric learning is often concerned with learning a pseudo-metric, e.g. in classification problems we would not necessarily want to make a distinction between objects of the same class, hence the distance between them is not constrained to be larger than zero.

Most metric learning methods learn a linear distance function. A linear distance function is limited to rotating and scaling the dimensions along which the distance is measured. Because of this it can also be computed by first applying a linear transformation to the input space and then measuring the Euclidean distance in the transformed space. This has the benefit that algorithms that rely on a Euclidean distance, such as nearest-neighbor look up with spatial indexing techniques, can be used off-the-shelf.

The distance function is often parameterized as a metric with a $d \times d$ parameter matrix $\mat{A}$:
\begin{equation}
d_{\mat{A}}(\vec{x}_i, \vec{x}_j) = \sqrt{(\vec{x}_i - \vec{x}_j)^T\mat{A}(\vec{x}_i - \vec{x}_j)}.
\label{eq:linearmetric}
\end{equation}
In the metric learning literature this form of distance function is often called a Mahalanobis distance, although this term was originally used for a specific instance of this distance, see Text Box \ref{tb:mahalanobis}. If $\mat{A}$ is positive definite then $d_{\mat{A}}$ is a metric and if $\mat{A}$ is positive semi-definite then $d_{\mat{A}}$ is a pseudo-metric. If $\mat{A}$ equals the identity matrix then $d_{\mat{A}}$ is simply the Euclidean distance.

\begin{textbox}
The Mahalanobis distance was introduced as a metric to measure distance between random vectors from the same distribution. \cite{mahalanobis1936generalised} The Mahalanobis distance is defined as:
\begin{equation}
d_{\textsc{Mahalanobis}}(\vec{x}_i, \vec{x}_j) = \sqrt{(\vec{x}_i - \vec{x}_j)^T\Sigma^{-1}(\vec{x}_i - \vec{x}_j)},
\end{equation}
where $\Sigma$ is the covariance of the distribution, or an estimate thereof. This metric is especially helpful for calculating the distance of a random vector to the mean of the distribution, since the metric ensures that for a Gaussian distribution the iso-surface corresponds to a constant probability of being generated by the distribution. Computing Mahalanobis distance between vectors is equivalent to calculating the Euclidean distance between whitened vectors. This can be easily seen by realizing that whitening data causes the covariance to be equal to the identity matrix, thus $\Sigma^{-1} = \mat{I}^{-1} = \mat{I}$, making the calculation for the Mahalanobis distance to be equal to Euclidean distance.
\caption{}
\label{tb:mahalanobis}
\end{textbox}

Because a Mahalanobis metric is a linear metric, it can be interpreted as the Euclidean distance in a linear transformation of the feature space. This is done by substituting $\mat{A} = \mat{G}^T\mat{G}$ and transforming points in the feature space by $\mat{G}$ in the following way:
\begin{align}
d_{\mat{A}}(\vec{x}_i, \vec{x}_j) & = \sqrt{(\vec{x}_i - \vec{x}_j)^T\mat{A}(\vec{x}_i - \vec{x}_j)}\nonumber \\
& =  \sqrt{(\vec{x}_i - \vec{x}_j)^T\mat{G}^T\mat{G}(\vec{x}_i - \vec{x}_j)} \nonumber \\
& =  \sqrt{\left(\mat{G}(\vec{x}_i - \vec{x}_j)\right)^T \mat{G}(\vec{x}_i - \vec{x}_j)} \nonumber \\
& =  \sqrt{(\mat{G}\vec{x}_i - \mat{G}\vec{x}_j)^T(\mat{G}\vec{x}_i - \mat{G}\vec{x}_j)} \nonumber \\
& =  \|\mat{G}\vec{x}_i - \mat{G}\vec{x}_j\|
\end{align}
Here $\mat{G}$ is a $r \times d$ matrix with $r$ being the rank of $\mat{A}$. \todo{Explain this better: first explain why this is a linear transformation by G, then explain why G shape is rxd, then explain fixed rank learning benefits.} Therefore, if the distance matrix $\mat{A}$ is not full-rank, transforming the points in feature space corresponds to projecting onto a lower-dimensional space $\mathbb{R}^r$. This can be useful for high-dimensional problems since it has the same effects as dimension reduction. Reducing the dimensionality of the problem gives a more compact feature representation and distance calculations are less expensive. Furthermore, high-dimensional spaces tend to be sparse as volume increases exponentially with the number of dimensions, often referred to as the curse of dimensionality. Hence, several metric learning methods try to learn low-rank metrics.

We can graphically represent a linear distance function as an ellipsoid in the original space. The ellipse then denotes the iso-contour of equal distance from the center of the ellipse. An example can be seen in Figure \ref{fig:approach_overview}. The axes of the ellipsoid correspond to the eigenvectors of the matrix $\mat{A}$ and the length of the axis should be equal to the corresponding eigenvalues to show the iso-line where the distance from the center of the ellipse is one. Hence, the Euclidean distance shown in the left of Figure \ref{fig:approach_overview} is a circle with radius one since the Euclidean distance corresponds to $\mat{A} = \mat{I}$. 



\subsection{Constraints}
\label{sec:background_constraints}

Metric learning methods require supervision in the form of constraints on the learned distance. When objects can easily be judged to be either similar or dissimilar, these can be given as a set of similarity/dissimilarity constraints. This is given as a set $\mathcal{S}$ of pairs $(i, j)$ that are similar and a set $\mathcal{D}$ of pairs that are dissimilar. These constraints have the following form:
\begin{subequations}
\begin{align}
d_{\mat{A}}(\vec{x}_i, \vec{x}_j) &\leq u & (i, j) &\in \mathcal{S}, \\
d_{\mat{A}}(\vec{x}_i, \vec{x}_j) &\geq \ell & (i, j) &\in \mathcal{D}.
\end{align}
\end{subequations}
They constrain the distances between similar pairs to be lower than some upper bound $u$ and distances between dissimilar pairs to be larger than some lower bound $\ell$. Suitable bounds have to be chosen for each problem. This is generally done by computing the 5th and 95th percentiles of the sampled distribution of distances between random pairs and equating these to  $u$ and $\ell$ respectively.

Another popular choice for distance constraints is relative constraints. Relative constraints constrain one distance to be larger or smaller than another distance. Relative constraints are defined as a set $\mathcal{R}$ of triplets of points $(i,j,k)$ and have the following form:
\begin{align}
d_{\mat{A}}(\vec{x}_i, \vec{x}_j) &< d_{\mat{A}}(\vec{x}_i, \vec{x}_k) & (i, j, k) &\in \mathcal{R}.
\end{align}
Thus for each triplet $(i,j,k) \in \mathcal{R}$ the distance between points $\vec{x}_i$ and $\vec{x}_j$ should be smaller than the distance between points $\vec{x}_i$ and $\vec{x}_k$.

Distance constraints are generated from the ground truth in the training set. For small problems with a small training set we can take a comprehensive approach and enumerate all possible pairs or triplets. However, for most real-world problems the training set has to be large and enumerating all pairs would be intractable. Therefore we have to resort to taking a sample of pairs from the training set. There are two main approaches to this sampling: global sampling and local sampling. Global sampling is the most general and corresponds to simply picking random pairs or triplets from the training set. This gives the most representative sample of distances in the feature space. However, many methods that use distance function in feature space are only concerned with pairs of points that have small pairwise distances. For example, \ac{kNN} only looks at the nearest neighbors of a point and any points that have a larger distance are ignored. In this case we can also restrict our samples to only consist of pairs or triplets that are contained in a local neighborhood: local sampling. One way to do this, which is specifically useful for \ac{kNN} applications is to sample a random point and then sample one or two random points amongst the nearest neighbors of this point in order to form a pair or triplet. Since the nearest neighbors depend on the distance function that is used they may change during the learning process. Therefore, we might want to iterate this sampling during the learning process, which we will refer to as iterative local sampling.



\subsection{Optimization and regularization}

While most linear metric learning methods use the same parametrize the distance function in the same way, as a Mahalanobis metric defined by (\ref{eq:linearmetric}), they differ in the way that they optimize the parameter matrix $\mat{A}$.

In \cite{kulis2012metric} the author present a general formulation of the metric learning problem that encompasses most metric learning methods. The problem consists of minimizing a loss function that includes a cost term and a regularization term and is given by:
\begin{equation}
\mathcal{L}(\mat{A}) = r(\mat{A}) + \lambda \sum_{i=1}^{m} c_i \left(\mat{X}^T\mat{A}\mat{X}\right).
\label{eq:metriclearningmodel}
\end{equation}
Here $r(\mat{A})$ is a regularization term which is a function of the parameter matrix $\mat{A}$, $c_1, ..., c_m$ are loss functions which encode the given constraints and $\lambda$ is a trade-off parameter between the regularization term and the costs. The way in which these cost functions are defined, the regularization function that is chosen and the way in which the resulting loss is optimized are the factors that distinguish the different linear metric learning methods.

In (\ref{eq:metriclearningmodel}) the cost functions are given as functions of $\mat{X}^T\mat{A}\mat{X}$ in order to be as general as possible. The term $\mat{X}^T\mat{A}\mat{X}$ denotes the pairwise distance matrix under the learned metric. This is simply there to show that the cost functions are evaluated on $d_\mat{A}$, the learned distance function. In practice the cost functions depend on the given constraints and generally only include the distances between the given pair or triplets of training points. For example, a popular choice is to encode similarity constraints as a hinge loss: 
\begin{align}
c(\mat{X}^T\mat{A}\mat{X}) &= \max(0,d_{\mat{A}}(x_i, x_j) - u)  &\forall (i,j) &\in \mathcal{S}, \\
c(\mat{X}^T\mat{A}\mat{X}) &= \max(0,\ell - d_{\mat{A}}(x_i, x_j)) &\forall (i,j) &\in \mathcal{D}.
\end{align}
And the equivalent formulation for a hinge loss function on relative distance constraints would be:
\begin{align}
c(\mat{X}^T\mat{A}\mat{X}) &= \max(0,d_{\mat{A}}(x_i, x_j) - d_{\mat{A}}(x_i, x_k)) &\forall (i,j,k) &\in \mathcal{R}.
\end{align}

Optionally, a regularization term is added to avoid over-fitting. Common examples are the Frobenius norm $\|\mat{A}\|_F$, which is the matrix equivalent to standard $L_2$ regularization and the trace norm $\tr(\mat{A})$ which is similar to $L_1$ regularization in that it prefers low-rank solutions.

The cost functions and regularization terms are generally chosen such that there is a tractable way to minimize the resulting loss function. This optimization can be cast as a minimization over the matrix $\mat{A}$ or by substituting $\mat{A} = \mat{G}^T\mat{G}$ it can be done as a minimization over $\mat{G}$. In the first case this results in a constrained optimization problem since $\mat{A}$ has to be \ac{PSD}. E.g. by reprojecting onto the cone of \ac{PSD} matrices during the optimization  by setting negative eigenvalues to 0. In the second case the problem is unconstrained since $\mat{G}^T\mat{G}$ will always yield a \ac{PSD} matrix, but will be in a quadratic form instead of a linear form.


%\subsection{Other approaches}

%local distance metrics

%non-linear distance metrics








\section{Related work}

\quad


\subsection{Linear metric learning}

%this section highlights differences between metric learning and our own work, for survey see Kulis etc.
There is a large body of work in the metric learning literature, therefore this thesis does not aim to provide a complete overview. Instead, this section highlights the differences between the main metric learning approaches and the work presented in this thesis. Readers looking for a recent survey are directed to \cite{kulis2012metric}, which discusses a wide body of metric learning approaches with reference to their unified model for regularized transformation learning, which we discussed in Section \ref{sec:background_opimization}. Readers looking for a more complete discussion of specific methods are directed to \cite{bellet2013survey} and \cite{yang2006distance}.


\subsubsection{Similarity/dissimilarity constraints}

%metric learning methods, mostly based on classification, uses similarity/dissimilarity constraints which can not capture a target metric (important point to explain)
As discussed in the introduction, most metric learning methods have been formulated in the context of a classification problem. This context determines the type of background knowledge that is available for training the metric. Hence, most popular metric learning methods use similarity/dissimilarity constraints, with pairs of inputs being constrained to be similar if they share the same classification label as output, and dissimilar if their output label differs. \cite{davis2007information, guillaumin2009you, kostinger2012large} Because these methods expect input pairs to be assigned to be either similar or dissimilar, they cannot be applied to a problem where similarity between pairs lies on a continuous scale. In comparison, the method presented in this thesis is specifically designed to align the learned metric to a target metric which can take any value. The method presented in this thesis can thus be applied to both binary similarity as well as continuous similarity.


\subsubsection{Relative distance constraints}

%metric learning methods with relative constraints, in theory able to encode target metric, would require a lot of triplets, not as applicable to large datasets, less strict supervision,
Metric learning methods that use relative distance constraints in the context of ranking problems, such as \cite{schultz2003learning}, are more similar to the work in this thesis since relative constraints can be used to indirectly encode a real-valued target metric. However, since these constraints do not directly encode the target distance they provide less supervision than the absolute distance constraints used by the method described in this thesis. Also, because of the large number of training triplets that would be needed to fully specify a target metric, these methods are less applicable to large datasets than the method presented in this thesis which relies on only a sampling of pairs.

%large margin methods, use relative constraints but in a discriminative fashion
Relative distance constraints are also used in the large-margin metric learning approach, as exemplified by \cite{weinberger2009distance, frome2007learning}, which has become popular because of good results on classification problems and its similarity to the well-known \ac{SVM} approach. Note however that because of the large-margin formulation these methods still depend on a binary similarity judgements for pairs of training points and are thus not applicable to problems with a real-valued target metric.


\subsubsection{Metric learning in structured prediction}

%metric learning for ranking \cite{mcfee2010metric}, learns distance function between query and train point, struct-SVM like solver that tries to satisfy as much of the training rankings as possible, satisfies specific rankings, not target metric (which would be between query and point)
By aligning the learned distance function to a real-valued loss function, the method described in this thesis is specifically suited for structured prediction problems which are characterized by such a loss function. In\cite{mcfee2010metric} metric learning is also applied in a structured prediction context and makes use of a real-valued loss function. The main differences with the approach described in this thesis is that McFee \& Lanckriet apply their method specifically to ranking problems and use a structural \ac{SVM} approach to learning. Since they specify their ranking problem as finding the best interleaving of relevant and irrelevant points, this method also requires binary similarity judgements. The loss function is only used to find the maximally violated constraints for the cutting-plane algorithm that they use for training. Therefore, this method is not applicable to learning from a target metric instead of from binary similarity judgements. \todo{Rewrite this paragraph/section since it is sooo messy.}


%metric learning methods for structured prediction \cite{guillaumin2009tagprop}
In \cite{guillaumin2009tagprop} metric learning is applied to the problem of image annotation. Like the work in this thesis, they use a nearest-neighbor approach to solving the problem, but instead of learning a distance function in feature space, they learn a weighted combination of base distance functions picked by hand. They optimize their weights by maximizing the log-likelihood for the training output, while this thesis takes the approach of directly approximating the target metric.

%attribute-based classification \cite{akata2013label}
This thesis applies metric learning to attribute-based classification. In \cite{akata2013label} label embedding is used as an approach to attribute-based classification. This can be viewed as learning a metric between input features and attribute vectors. In comparison, this thesis presents a method that learns a metric between pairs of input vectors which is aligned to the loss between attribute vectors. Thus, while this thesis takes a pure metric learning approach, \cite{akata2013label} takes a label embedding, or scoring function, approach.

%\cite{bellet2013survey} surveys several methods for dealing with structured input data, note that these methods deal with structured input data, while we deal with a regular feature space, but in combination with a target metric derived from structured output data with a continuous loss function, the surveyed metric learning methods still learn from similarity/dissimilarity pairs and are thus not applicable to a target metric
Note that, although \cite{bellet2013survey} investigates the application of metric learning to structured data, they focus on a structured input space, while the work in this thesis focuses on a structured output space. The methods discussed in \cite{bellet2013survey} thus still rely on binary similarity/dissimilarity constraints.

\subsubsection{Metric learning as regression}

%\cite{lowe1995similarity} metric learning for kNN classification
The idea of learning a metric to improve \ac{kNN} classification is not new. In \cite{lowe1995similarity} a weighted Euclidean metric is learned through conjugate gradient descent on a squared prediction error measure. Although this approach is similar in solution to the work in this thesis, the target is very different. Like most metric learning methods, this approach relies on the classification problem consisting of a set of discrete labels.

%metric learning for regression \cite{weinberger2007metric} comes very close to what we do but is specific to the regression problem,
The method in \cite{weinberger2007metric} learns a metric that minimizes regression error for a kernel regression algorithm. Since they formulate their error as the quadratic prediction error and minimize this using \ac{SGD}, they have a very similar derivation of the learning problem as the work presented in this thesis. However, Weinberger \$ Tesauro only apply their method directly to the kernel regression problem. Thus in their method the target metric is directly the function to be learned, while the method in this thesis uses the target metric to learn a representation of the input features that allows \ac{kNN} to better solve a structured prediction problem by aligning the learned metric to the structured loss function.

%\cite{meyer2011regression} formulates a regression method to learn a metric which they apply to similarity constraints and kernel learning, we use the same formulation but apply it to learning from a target metric
The regression problem for metric learning from a squared prediction error is further investigated in \cite{meyer2011regression}. The work in this thesis uses exactly the same formulation for the \ac{SGD} problem as Meyer et al. and they further show fixed-rank and scale invariant variants, which are applicable as solving methods for the work in this thesis. However, in \cite{meyer2011regression} this method is only applied to a classical classification metric learning problem with similarity/dissimilarity constraints. The work in this thesis goes further than that and describes how this method can be applied to solve metric learning problems for structured prediction problems with absolute distance constraints.


\subsection{MDS and related techniques}

%MDS, sammon mapping, isomap, LLE, SNE supervised and uses a distance matrix to learn the new representation, used primarily for visualisation, transductive learning that does not generalize, while we operate in inductive paradigm and our method can be applied to unseen test points
Like the method in this thesis, \acf{MDS} and related techniques also use target distances as a training signal. This set of techniques includes \ac{MDS} \cite{venna2006local, chen2009local, chen2013stress}, kernel PCA \cite{scholkopf1997kernel}, isomap \cite{tenenbaum2000global}, and others. What they have in common is that they take as input a distance matrix, which is comparable with our input of a target metric which could also be given as a distance matrix. However, all these methods learn an embedding, or mapping, of the training points in a different space. Thus instead of solving for the parameters of the distance function, they solve for the point coordinates. Because of this, these methods are very applicable to dimension reduction and visualization, but not to metric learning.

%MDS stress functions significantly change the characteristics of the method, while we use only squared prediction error, other stress functions might also be included in our method 
\cite{chen2013stress} discusses a family of optimization functions, called stress functions, for \ac{MDS} methods. The squared error function that this thesis uses corresponds to the most naive stress function and \cite{chen2013stress} show that by changing the stress function the characteristics of the solution can be tailored to a specific problem. Thus, finding a method to minimize these stress functions with respect to the metric parameters instead of to the point coordinates might change the characteristics of the method presented in this thesis and make it more broadly applicable.



\subsection{Representation learning}

%representation learning, unsupervised, focusses on properties of the learned representation such as meaningfulness, invariance, etc, but not on distances
The method presented in this thesis learns a feature transformation which makes distances in feature space more aligned with a target metric, thus resulting in a feature representation which makes learning problems that are characterized by that target metric easier to solve. According to a recent survey, the problem of representation learning is ``learning representations of the data that make it easier to extract useful information when building classifiers or other predictors''. \cite{bengio2013representation} These descriptions seem very similar. However, the field of representation learning deals with unsupervised methods that learn a completely new representation that can be considered helpful in a general way, while this thesis presents a method that adapts a given representation in a supervised, task-specific way.



%\subsection{Kernel learning}

%structured prediction methods with kernels generally use fixed kernels, in kernel learning either the parameters of a fixed kernel are learned or a kernel matrix in a transductive setting is learned, while we learn a full metric directly on the input space



%\subsection{Instance-based structured prediction}

%structured random forest \cite{kontschieder2011structured}, trees work as a nearest neighbor index, put examples close together if he have small entropy in labels, however uses distinct labels for entropy, while we use the target metric between the whole structured output which is also applicable to other structured problems that do not have distinct labels

%kernels for structured prediction SVM

\chapter{Metric alignment}

%intuition, align feature space to ground truth space, real-valued ground-truth distances, loss function in structured prediction problems, minimization of prediction error by regression
Our metric alignment method is based on the intuition that, if the ground truth in our training data has a structure on which we can define a real-valued loss function, we can use this loss function to inform our metric learning.
Specifically, we learn a metric that minimizes the difference between the distance in feature space and the loss between the corresponding ground-truth output structures.
The learned metric thus makes distances in feature space predictive of loss in ground-truth space.\footnote{Note that the output structure need not necessarily induce any specific type of space, all we need is a positive, real-valued loss function defined on a large enough sample of pairs from the training set.}
This predictive power is especially useful when applying nearest neighbor methods to structured prediction problems.



\section{Target metric}

%target metric, $\hat{d}$ mapping from pairs of indexes to non-negative real number
The metric alignment method takes as input a \emph{target metric} and a set of \emph{absolute distance constraints}.
The target metric is a function $\hat{d}$ which assigns a distance to pairs $(i,j)$ of input points.
For our applications we define the target metric to be equal to the loss $\ell$ for prediction output $\vec{y}_j$ when the correct output is $\vec{y}_i$:
\begin{equation}
\hat{d}(i,j) = \ell(\vec{y}_i, \vec{y}_j).
\end{equation}

%preferably a squared metric function, but non-metric functions can also be approximated
Although we use the structured loss as target metric, the notion of a target metric is more general than that.
In fact, the target metric can be any non-negative value as long as it is known for a number of training pairs much larger than the number of dimensions of the feature space.
It could even be background knowledge provided by hand.
However, since it will be approximated by a pseudo-metric function, it needs to be non-negative and the approximation will be better for target metrics that themselves are of a pseudo-metric nature.

\section{Absolute distance constraints}
%absolute distance constraints, encode target metric %if target metric is non-symmetric, both directions should be a constraint
Absolute distance constraints are the equivalent of similarity/dissimilarity constraints that use a target metric instead of upper or lower bounds.
Absolute distance constraints are given as a set $\mathcal{A}$ of pairs $(i,j) \in \mathcal{A}$, each associated with a ground-truth distance given by the target metric. They are thus defined as follows:
\begin{align}
d_{\mat{A}}(\vec{x}_i, \vec{x}_j) &= \hat{d}(i,j) & (i, j) &\in \mathcal{A}.
\end{align}
Here $d_{\mat{A}}(\vec{x}_i, \vec{x}_j)$ is the learned metric evaluated on the pair of points in feature space $(\vec{x}_i, \vec{x}_j)$ that corresponds to the pair $(i,j) \in \mathcal{A}$.
And $\hat{d}(i,j)$ is the value of the target metric corresponding to that pair.
The set of pairs $\mathcal{A}$ can be sampled in the same way as the set of pairs $\mathcal{S}$ in the case of similarity constraints.
Note however that if the target metric is non-symmetric, the pairs should be treated as ordered pairs and preferably both $(i,j)$ and $(j,i)$ should be included in the set.



\section{Loss function}

%squared error loss function
In order to learn a metric from absolute distance constraints we minimize the quadratic prediction error on the target distances given by the constraint. Thus our loss function corresponds to the \acf{MSE} over the training set:
\begin{equation}
\text{MSE}_\mathcal{A}(\mat{A}) = \frac{1}{|\mathcal{A}|}\sum_{(i,j) \in \mathcal{A}} \left( \hat{d}_{ij} - d_{\mat{A}}(\vec{x}_i, \vec{x}_j) \right)^2.
\label{eq:mse}
\end{equation}
Minimizing the \ac{MSE} over the training pairs corresponds to minimizing the empirical risk under a quadratic loss function.


\section{Convex optimization using stochastic gradient descent}

%minimization,  linear transform of input space A = G^T G, solving for G means unconstrained optimization problem, convex for full rank G, large, SGD, gradient, batch gradient, learning rate, parameter selection, convergence speed, ASGD, learning rate, mini-batches, hyper parameters,
In order to fit the transformation matrix to the training set we use \acf{SGD}. Gradient descent is an iterative first order optimization algorithm that finds a local minimum of an error function. On each iteration it changes the parameters by a small step in the direction opposite to the gradient of the error function with respect to the parameters. Since the negative gradient is the direction of steepest descent, each step leads to a reduction in error. When the error function is defined as a sum of individual errors on examples in a training set, \ac{SGD} is a stochastic version of gradient descent where at each step the gradient is computed on the error of a single example. This does not require a summation over all examples in the training set for each update, which is beneficial when the training set is large.

In order to apply \ac{SGD} to our learning problem we first define our metric using the decomposition $\mat{A} = \mat{G}^T\mat{G}$ to remove the \ac{PSD} constraint on $\mat{A}$ and turn the problem into a unconstrained optimization problem. Further, we use squared distances to remove the square root to further simplify the gradient derivation. Our distance function is thus defined as:
\begin{align}
d_{\mat{A}}(\vec{x}_i, \vec{x}_j)^2 &= (\vec{x}_i - \vec{x}_j)^T \mat{A} (\vec{x}_i - \vec{x}_j) \\
&= (\vec{x}_i - \vec{x}_j) \mat{G}^T \mat{G} (\vec{x}_i - \vec{x}_j) \\
&= \|\mat{G}(\vec{x}_i - \vec{x}_j)\|^2
\end{align}
We then define an error function which has the same minimum as (\ref{eq:mse}) but is easier to differentiate and optimize because of its form:
\begin{equation}
E_{\mathcal{A}}(\mat{G}) =  \sum_{(i,j) \in \mathcal{A}}  \frac{1}{4} \left( \hat{d}(i,j)^2 -  \|\mat{G}(\vec{x}_i - \vec{x}_j)\|^2 \right)^2.
\label{eq:opt_target}
\end{equation}
Here $\vec{x}_i - \vec{x}_j$ is the pairwise difference between two vectors in the input space $\mathcal{X}$ and $\hat{d}(i,j)^2$ is the squared target distance for this pair of input vectors. And $\|\mat{G}(\vec{x}_i - \vec{x}_j)\|^2$ is our learned squared metric, which is efficient to compute as the inner product of the transformed difference vector with itself. The constant vector $\frac{1}{4}$ is there only for convenience in differentiation. 

Interpreting the optimization objective as $E_{\mathcal{A}}(\mat{G}) = \sum_{(i,j)\in\mathcal{A}} E_{ij}(\mat{G})$ the corresponding gradient of the error $E_{ij}$ with respect to $\mat{G}$ on a single pairwise distance constraint $(i,j) \in \mathcal{A}$ is given by:
\begin{align}
\nabla E_{ij}(\mat{G})  &= \frac{\delta}{\delta \mat{G}} \frac{1}{4} \left( \hat{d}(i,j)^2 -  \|\mat{G}(\vec{x}_i - \vec{x}_j)\|^2 \right)^2 \\
&= \frac{1}{2} \left( \hat{d}(i,j)^2 -  \|\mat{G}(\vec{x}_i - \vec{x}_j)\|^2 \right) \frac{\delta}{\delta \mat{G}} \left( \hat{d}(i,j)^2 -  \|\mat{G}(\vec{x}_i - \vec{x}_j)\|^2 \right) \\
&= -\left( \hat{d}(i,j)^2 -  \|\mat{G}(\vec{x}_i - \vec{x}_j)\|^2\right) \mat{G} (\vec{x}_i - \vec{x}_j)(\vec{x}_i - \vec{x}_j)^T.
\label{eq:opt_grad}
\end{align}

We initialize our transformation as the identity transformation $\mat{G} = \mat{I}$ and then update it at each iteration as:
\begin{equation}
\mat{G} \leftarrow \mat{G} - \eta_{t} \nabla E_{ij}(\mat{G}).
\label{eq:update}
\end{equation}
Here $t$ is the iteration number, $\nabla E_{ij}(\mat{G})$ is the gradient computed over the distance constraint for points $(i,j) \in \mathcal{A}$ and $\eta_{t}$ is the learning rate: a gain parameter which controls the step size. The learning rate should decrease and the schedule with which it decreases is important for the convergence speed of the optimization \cite{xu2011towards}. We calculate $\eta_{t}$ at each time step $t$ as follows:
\begin{equation}
\eta_{t} = \frac{\eta_0}{\left(1+ \eta_0 t \right)},
\label{eq:eta_update}
\end{equation}
where $\eta_0$ is a hyper parameter that we set by 2-fold cross validation on the training set.






%%%%%%%%%%%%%%%%%%%%%%%%%%%%%%%%%%%%%%%%%%%%%%%%%%%%%%%%%%%%%%%%%%%%%
%
% EXPERIMENTS
%%%%%%%%%%%%%%%%%%%%%%%%%%%%%%%%%%%%%%%%%%%%%%%%%%%%%%%%%%%%%%%%%%%%%


\section{Results}

To evaluate metric alignment we need a dataset for which a real-valued target metric is defined. We took our testing data from a semantic segmentation benchmark. Semantic segmentation is a computer vision problem in which the objective is to segment input images into contiguous regions by assigning labels to each pixel from a predefined set of class labels. 


Segmentation benchmarks consist of images $\mat{I}$ where each pixel $\mat{I}_{(i,j)}$ is usually given as a triple $(r,g,b)$ denoting the color of a single pixel and segmentations $\mat{S}$ of which each element  $\mat{S}_{(i,j)} \in \mathcal{L}$ assigns a label to pixel $\mat{I}_{(i,j)}$ of the image. Here $\mathcal{L} = \{1, 2, \ldots, n\}$ is a pre-determined label set  where each label corresponds to one of $n$ visual classes. For our dataset we extract square patches from the images and the corresponding segmentations.


\begin{equation}
L(\vec{y}, \vec{y}^\prime) = 1 - \frac{\sum_{i,j} \left [ y_{i,j} = y^\prime_{i,j}\right]}{\sum_{i,j} \left [y_{i,j}, y^\prime_{i,j} \in \mathcal{L}\right ]}.
\label{eq:patch_loss}
\end{equation}



\section{Application to semantic segmentation}




%semantic segmentation, per-pixel classification, apply \ac{kNN} to image patches, patch has feature descriptor and label patch, labels correspond to object classes, loss is distance in label space, align feature space to label space, improved segmentation accuracy

%semantic segmentation, challenging problem, combines object recognition and image segmentation, applications in image understanding, image retrieval
Semantic segmentation is a challenging problem that combines object recognition and image segmentation, with applications in image understanding and image retrieval. The objective is to segment input images into contiguous regions by assigning labels to each pixel from a predefined set of class labels, in effect performing a per-pixel classification of the image. Figure \ref{fig:segmentation_examples} shows some example input and output for the semantic segmentation task and we give a more formal description below.

\begin{quotation}
\subsubsection*{Semantic segmentation problem.}

\paragraph{Input:} An image $\mat{I}$ where each pixel $\mat{I}_{(i,j)}$ is usually given as a triple $(r,g,b)$ denoting the color of a single pixel. And a label set $\mathcal{L} = \{1, 2, \ldots, n\}$ where each label corresponds to one of $n$ pre-determined object classes.

\paragraph{Output:} A segmentation image $\mat{S}$ of which each element  $\mat{S}_{(i,j)} \in \mathcal{L}$ assigns a label to pixel $\mat{I}_{(i,j)}$ of the image.

\paragraph{Learning problem:} given a set of training images $\mathcal{I}_\textsc{train} \subset \mathcal{I}$ with the corresponding ground truth segmentations $\mathcal{S}_\textsc{train} \subset \mathcal{S}$ and a loss function $L: \mathcal{S} \times \mathcal{S} \rightarrow \mathbb{R}$, learn a function $h(\mat{I}) = \mat{S}$ that minimizes this loss over unseen images. %\todo{...minimizes this loss over true distribution of images.}
\end{quotation}


Because the output consists of a 2-dimensional label array this is a structured prediction problem. We solve this problem by combining \ac{kNN} classification with metric alignment. First we divide images into overlapping patches and compute a feature vector for each patch. We use metric alignment to learn a transformation of these feature vectors that aligns distances in feature space with loss in label space. Then we classify each patch using structured \ac{kNN} classification.

\begin{figure}[tbh]
\begin{center}
\missingfigure{Top row: image from MSRC, VOC, CamVid. Bottom row: corresponding segmentations.}
\caption{Some examples of the semantic segmentation task. Top row: input images from the MSRC, Pascal VOC, and CamVid benchmarks. Bottom row: the manual segmentations corresponding to the input images. Each color denotes a different object class. Note the difference in annotation style between the benchmarks.}
\label{fig:segmentation_examples}
\end{center}
\end{figure}

% common approaches, structured learning CRF, tree ensembles with context features, segmentation with object detectors, clustering
%The main challenges in semantic segmentation are a combination of the challenges found in object recognition, image segmentation, and structured prediction. Like in object recognition there is high variance in appearance, orientation and shape of objects belonging to object classes of interest. However, since single pixels have to be labeled, there is less information available for making a labeling decision. Semantic segmentation methods solve this either by pre-segmenting the image into superpixels or larger segment hypotheses such that evidence, e.g. object recognition response, can be accumulated over a larger area, or by using features which aggregate information about a larger area around the pixel, its context. Like in structured prediction problems, maintaining consistency between predicted labels is also an important challenge. Because of this, many semantic segmentation methods use a \ac{HCRF} approach which allows the penalization of inconsistency on multiple levels unless there is strong evidence for non-continuity in the labeling.\todo{Give some global background describing challenges of segmentation (much variance in appearance and shape, label consistency, finding correct scales for segmentation) and the main approaches to solving them (HCRF, superpixels, SVM, which differ by benchmark).}\todo{Segmentation box? Show different segmentation tasks and talk about basic methods like superpixels and clustering methods?}

\subsection{Structured \acl{kNN} segmentation}

%segmentation with kNN, kNN is simple, requires little parameter tuning, and is applicable to multi-class problems, 
Our segmentation method is based on \acf{kNN} classification. \ac{kNN} is a non-parametric classification method, which means that its decision function is not characterized by a parameterized model. Instead, the class for an input point is decided to be the majority class amongst the most similar points in the training set, the nearest neighbors. Although it requires storing of the whole training set, which may become problematic for large sets, the \ac{kNN} classification method is simple, requires little parameter tuning and is inherently applicable to multi-class problems. \todo{Box: the k-nearest neighbor classification?}

%we use a simple generalization to structured prediction, this is patch based, find nearest patches and interpolate each label of the patch independently
We generalize \ac{kNN} to the case of structured prediction by applying it to structured label patches. %, following the idea in \cite{kontschieder2011structured} of generalizing random forests to use structured label patches. 
We define a structured output space $\mathcal{Y} = \mathcal{L}^{d \times d}$ consisting of label patches $\vec{y}$ whose elements $y_{ij} \in \mathcal{L}$ each consist of one of the pre-determined class labels. When predicting the label patch $\hat{\vec{y}}$ for a given input vector we look up the label patches $\mathcal{Y}_\textsc{nn}$ of the $k$ nearest neighbor vectors  and interpolate each label position independently based on the labels at the same position $(i,j)$ of the nearest neighbor patches:
\begin{equation}
\hat{y}_{ij} = \argmax_{\ell \in \mathcal{L}} \sum_{\vec{y} \in \mathcal{Y}_\textsc{nn}} \left[ y_{ij} = \ell \right ].
\label{eq:nn_decision_rule}
\end{equation}
Here $\hat{y}_{ij}$ is the pixel at position $(i,j)$ of the output patch $\hat{\vec{y}}$ and $\left [\ \cdot \ \right ]$ denotes the Iverson bracket which is equal to 1 if the enclosed statement is true and 0 otherwise. This computation corresponds to letting each pixel of each neighbor patch vote on the class of the corresponding output pixel and then taking the most voted class for that output pixel.

\begin{figure}[tbh]
\begin{center}
\missingfigure{Figure explaining searching for nearest neighbor vectors in feature space and patch creation from neighbor patches according to equation \ref{eq:nn_decision_rule}}
\caption{Schematic overview of the \ac{kNN} algorithm with structured label patches. }
\label{fig:knn_segmentation}
\end{center}
\end{figure}

%for each of the training images extract $d \times d$ image patches from $I$ on regular grid $(\vec{u}, \vec{v})$, compute array of image features $X = \left [ \vec{x}_1, \vec{x}_2, \ldots, \vec{x}_p \right ]$ with $\vec{x}_i$ computed from the image patch centered on image coordinates $(u_i, v_i)$, use SIFT and CSIFT features, also extract label patches from the corresponding image labeling $L$ such that label patch  $\vec{y}_i$ corresponds to the $d' \times d'$ label patch around $(u_i, v_i)$, store these coordinates and create index for the features such that we can efficiently find the closest points
From each training image $\mat{I}$ we extract feature vectors $\mathcal{X}_\mat{I} = \left\{ \vec{x}_1, \vec{x}_2, \ldots, \vec{x}_p \right\}$ on a regular hexagonal grid $(\vec{u}, \vec{v})$, with $\vec{x}_i$ computed from the $d^\prime \times d^\prime$ image area centered on image coordinates $(u_i, v_i)$. Each of these feature vectors $\vec{x}_i$ is paired to a label patch $\vec{y}_i$ of size $d \times d$ taken from the training annotation for the image and centered at $(u_i, v_i)$. Note that the size $d$ for the label patch and $d^\prime$ for the feature patch can be different. Specifically, $d^\prime$ is dependent on the scale of the image features while $d$ is a hyper-parameter that we set. For convenience we define these patches to have the origin $y_{0,0}$ in the center of the patch, such that the subscripts $i$ and $j$ run from $-\tfrac{d}{2}$ to $\tfrac{d}{2}$. Together these feature vectors form a training set $\mathcal{X}_\textsc{train} = \bigcup\left\{\mathcal{X}_\mat{I}: \mat{I} \in \mathcal{I}_\textsc{train} \right\}$ We store all the training vectors and label patches and we create a randomized kd-trees index for the feature vectors using FLANN \cite{muja2009fast} such that we can efficiently find the closest points from this set given an input vector.

%for each test image, again compute features for each patch, for each feature find the closest patch in the training set, extract label patch from the same area, merge overlapping label patches to create the output labeling
For each test image $\mat{I}$ we again extract feature vectors $\mathcal{X}_\mat{I}$ on the grid $(\vec{u}, \vec{v})$. For each of these feature vectors we search for the nearest neighbors amongst the training vectors and their associated label patches. From these we then generate the hypothesis patches as per (\ref{eq:nn_decision_rule}) giving us a label patch centered at each grid point $(u_i, v_i)$. These hypothesis patches are partially overlapping. Thus for each pixel in the image we select the most voted class among the positions in patches that overlap that pixel, similar to (\ref{eq:nn_decision_rule}).

%\begin{algorithm}
%\caption{The structured k-nearest neighbor algorithm for label patches.}
%\label{alg:structured_kNN}
%\begin{algorithmic}[1]
%\REQUIRE  $\mat{X}_\textsc{train}$, $\mat{Y}_\textsc{train}$, $\mat{X}_\mat{I}$ 
%\ENSURE $S = h(\mat{I})$
%\STATE TODO: give clear pseudocode
%\end{algorithmic}
%\end{algorithm}

\subsection{Aligning distances in feature space to loss}

%effectiveness of \ac{kNN} method highly dependent on representation function, meaning close in feature space has to align with semantically close, if appearance and semantics are not aligned this will lead to wrong prediction based on nearest neighbors
The effectiveness of the structured \ac{kNN} method described above is dependent on both the representation function and the distance function. The  distance function defines the way in which distances are measured in feature space, while the representation function defines the feature space. In order to be effective, our method needs semantically similar points to lay close together in feature space, with `semantically similar' meaning that the corresponding label patches have a large overlap.

%use metric alignment to learn feature transform, label patches, loss defines distance in label space, we want small distance to mean small loss
The naive approach is to use Euclidean distance as a default and then to search for features which make similar patches lie close together. However, it is very difficult to engineer such an image feature. Instead, we will use existing features which have been applied in many image classification tasks and use the metric alignment method described in section \ref{metric_regression} to learn a distance metric which makes distance between feature vectors predictive of loss between the corresponding labels. We define this loss as a continuous function $L: \mathcal{Y} \times \mathcal{Y} \rightarrow \mathbb{R}$ with range $[0,1]$ which gives us an inverse of the overlap between label patches:
\begin{equation}
L(\vec{y}, \vec{y}^\prime) = 1 - \frac{\sum_{i,j} \left [ y_{i,j} = y^\prime_{i,j}\right]}{\sum_{i,j} \left [y_{i,j}, y^\prime_{i,j} \in \mathcal{L}\right ]}.
\label{eq:patch_loss}
\end{equation}
Here $L(\vec{y}, \vec{y}^\prime)$ is the loss for predicting patch $\vec{y}^\prime$ for patch $\vec{y}$. This loss counts the positions that have been assigned the same label in both patches as a fraction of the positions that been assigned a label in both patches, since some positions are not assigned a label during annotation. Since the overlap is a similarity measure, we turn it into a distance measure by subtracting it from 1. By training the distance metric on this pairwise loss, we try to align the feature space to the label space in such a way that when we search for nearest neighbors in feature space we will find neighbors that have label patches that are close to the correct label patch for the input feature vector.


%%%%
%%%%
%%%%


\subsection{Experimental setup}

%We evaluate our method by applying our segmentation pipeline to benchmark problems with Euclidean distance and with distance metric learned by our metric alignment method.
%learning to rank, LMNN, RF (Kontschieder et al.)
To evaluate our method we perform an experiment using two semantic segmentation benchmarks. In this experiment we apply the \ac{kNN} segmentation method described above using a distance function learned by applying metric alignment on the training data. We compare this against the result of applying the \ac{kNN} method using standard Euclidean distance.

%We compare our method against two baseline large-margin metric learning methods: LMNN \cite{weinberger2009distance} which learns from discrete examples and \cite{schultz2003learning}
Besides the Euclidean baseline we also compare against two alternative metric learning methods, both of which use a large margin approach to metric learning. First against \ac{LMNN} \cite{weinberger2009distance}, described in section \ref{discrete_metrics}, and secondly against \cite{schultz2003learning}, described in section \ref{ranking_metrics}. We use \ac{LMNN} as a baseline because it is a well tested method designed specifically to improve \ac{kNN} classification results. The second method was chosen because it is similar in spirit to our own approach in that it learns from relative comparisons, but it is different in practice since it uses qualitative training data while metric alignment uses quantitative data.

%We evaluate both the accuracy of the learned metrics on pairwise differences and the effect of the metric learning on the segmentation task using our \ac{kNN} segmentation method.
To compare the different methods in this experiment we look at how well the distance function between image patches approximates the label distance as measured by \ac{MSE} as well as the overall increase in segmentation accuracy as measured by the benchmark evaluation measures.



\subsubsection{Datasets}

%We evaluate our method on the MSRC and VOC segmentation datasets; two benchmarks in semantic segmentation.
To evaluate our methods we use two datasets of images which have been manually segmented into  \footnote{\url{http://research.microsoft.com/en-us/projects/objectclassrecognition/}}

The MSRC set consists of a small number of images, loosely segmented into 21 classes. covering most of each image with little variation in appearance and scene.

Camvid dataset

The VOC set consists of a large number of images with strict segmentation into 20 foreground classes and an aggregate background class with much variation in appearance. %Object detection challenge, hence background class, but this is challenging for segmentation methods because you have to decide what not to segment.

Difference in annotation styles, MSRC is loosely segmented while the Camvid and VOC are close to pixel-perfect. VOC generally has one or a few foreground objects segmented while the rest is grouped as background, while MSRC and CamVid are fully segmented with background classes being labeled, e.g. `grass', `street'. See figure \ref{segmentation_examples}.

VOC is more challenging than MSRC and CamVid on both effectiveness and sheer size.


\subsubsection{Implementation}

We use SIFT and C-SIFT image descriptors as feature vectors extracted using the ColorDescriptor software. \cite{sande2011empowering} We then index these training vectors using the multiple kd-tree method from the FLANN library \cite{muja2009fast}, which allows us to efficiently calculate nearest neighbors amongst these points given an input vector.

\todo{Box: image descriptors?}

\subsubsection{Training and parameter selection}

We divide the dataset into non-overlapping training and testing images. 


extract training pairs from training images, note: can't use all patch pairs because this are intractably many, we use a set of randomly selected points and their k-nearest neighbors

train on feature vectors and pairwise loss, compute gradient over the neighborhood set at each step

\todo{Explain tuning.}






\subsubsection{Evaluation}



We measure the \ac{MSE} between the label distance and the learned nearest-neighbor on a sample of testing pairs

%nearest neighbor MSE histogram?

we evaluate global pixel accuracy and average class pixel accuracy as well as the intersection over union measure used in the pascal VOC dataset

we show per patch results for different numbers of neighbors 

plot of label distance vs learned metric distance, both normalized\

plot of segmentation accuracy as a function of neighbors?

plot of distance to closest ground-truth patch?





%%%%
%%%%	
%%%%








\subsection{Results}

%Training curve with MSE on training pairs

%MSE on patch overlap for testing pairs

%effect on segmentation score

%effect of features and dimension reduction on learnability

%effect of parameters for minimization, not all training schemes seem to work





We show that learning a simple linear transformation improves alignment between pairwise Euclidean distance in feature space and pairwise loss, which improves overall segmentation accuracy.




































%%%%%%%%%%%%%%%%%%%%%%%%%%%%%%%%%%%%%%%%%%%%%%%%%%%%%%%%%%%%%%%%%%%%%
%
% CONCLUSIONS
%%%%%%%%%%%%%%%%%%%%%%%%%%%%%%%%%%%%%%%%%%%%%%%%%%%%%%%%%%%%%%%%%%%%%

\pagebreak
\section{Conclusions}

%list the specific cases/requirements in which this is useful (see intro)















\bibliographystyle{plain}
\bibliography{Thesis}


%\appendix
%\section{Glossary}

%\begin{description}
%\item[Similarity measure] A measurement of the similarity between two objects. 
%\item[Distance function] A function which maps pairs of points in some space to a positive real number or zero.
%\item[Metric] See \emph{distance function}.
%\item[Metric transformation] A linear transformation of a feature space for which it holds that Euclidean distances in the transformed space correspond to a specified distance function on the original space. 
%\end{description}

\end{document}  